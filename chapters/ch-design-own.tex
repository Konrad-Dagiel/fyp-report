\chapter{The Interstellar Problem}
The robot TARS has been sucked into the singularity, an N-dimensional grid. He starts in a corner of the grid. In order to escape, he must reach the opposite corner of the singularity from where he started.
Inside, he can move in any direction through any dimension, as long as he is making progress towards his escape point with each move.
\begin{enumerate}
    \item How many different paths could TARS possibly take through the singularity in order to reach his goal and escape?
    \item If each space in the singularity has a fuel cost, what is the minimum cost path from the start position to the goal?
\end{enumerate}

\section{The Interstellar Problem 1}
\begin{description}
    \item[Problem Statement:]
    Given as input the shape of an n-rank tensor, find the number of unique paths from a corner of the tensor denoted $R$ to the opposite corner of the tensor,
    denoted $F$, where the only moves allowed at any point in time strictly make progress towards $F$.
        
    \item[Input:]
    A tuple of integers $dimensions$, where $|dimensions|$ is the number of dimensions in the n-rank tensor, and each integer in $dimensions$ represents the size of the corresponding dimension.
        
    \item[Output:] 
    An integer $paths$, which represents the number of unique paths (where negative progress is not allowed) from $R$ to $F$.
        
    \item[Example 1:] For:\\
    $dimensions$ = $(10)$\\
    $paths = 1$
    \item[Explanation 1:]
    There is only one dimension in the singularity, hence one path:
    \begin{table}[H]
        \centering
        \begin{tabular}{|c|c|c|c|c|c|c|c|c|c|}
            \hline
            \textbf{R} & $\rightarrow$ & $\rightarrow$ &$\rightarrow$ & $\rightarrow$ & $\rightarrow$& $\rightarrow$&$\rightarrow$ & \textbf{F} \\
            \hline
        \end{tabular}
    \end{table}

    \item[Example 2:] For:\\
    $dimensions$ = $(3,7)$\\
    $paths = 28$

    \item[Explanation 2:]
    In this case the singularity has two dimensions, and the shape is $(3,7)$. This means TARS can go right or down, as any other movement would be negative progress.
    There are 28 unique ways the robot can reach $F$ from $R$.
    Example Path:

    \begin{table}[htbp]
        \centering
        \begin{tabular}{|c|c|c|c|c|c|c|}
            \hline
            \textbf{R} & $\rightarrow$ & $\downarrow$ &  &  &  &  \\
            \hline
             &  & $\rightarrow$ & $\rightarrow$ & $\rightarrow$ & $\rightarrow$ & $\downarrow$ \\
            \hline
             &  &  &  &  &  & \textbf{F} \\
            \hline
        \end{tabular}
    \end{table}

    \item[Example 3:] For:\\
    $dimensions$ = $(2,7,5,2)$\\
    $paths = 27720$

    \item[Explanation 3:]
    In this case the singularity has 4 dimensions, making it impossible to visualize. There are 27720 different paths TARS could take to reach $F$ from $R$.
\end{description}

\section{The Interstellar Problem 2}
\begin{description}
    \item[Problem Statement:]
    Given as input an n-rank tensor where the value at each cell represents the fuel cost of landing on that cell, find the cost of the path from a corner of the tensor denoted $R$ to the opposite corner of the tensor,
    denoted $F$ with the minimum total fuel cost, where the only moves allowed at any point in time strictly make progress towards $F$.
        
    \item[Input:]
    An n-rank tensor of integers called $cost$, where each cell in $cost$ represents the fuel cost of landing on that cell.
        
    \item[Output:] 
    An integer $min\_path$, which represents the cost of the cheapest path (where negative progress is not allowed) from $R$ to $F$.
        
    \item[Example 1:] For:\\
    $cost$ = $[10]$\\
    $min\_path = 10$
    \item[Explanation 1:]
    The singularity is a single cell of cost 10, so the total cost of the minimum path is 10:

    \item[Example 2:] For:\\
    $cost = $\\
    $[[1,3,1],$\\
    $[1,5,1],$\\
    $[4,2,1]]$\\
    $min\_path = 7$

    \item[Explanation 2:]
        Explanation: $1 + 3 + 1 + 1 + 1 = 7$ is the minimum path sum.
        \begin{table}[H]
            \centering
            \begin{tabular}{|c|c|c|}
                \hline
                \textbf{R} & $\rightarrow$ & $\downarrow$ \\
                \hline
                 &  & $\downarrow$ \\
                \hline
                 &  & \textbf{F} \\
                \hline
            \end{tabular}
        \end{table}
\end{description}

\section{Building Up to The Interstellar Problem}
In order to understand the solution to the interstellar problem, we must first understand the following smaller problems,
and build up to the solution to the interstellar problem.

\subsection{Unique Paths}
\begin{description}
    \item[Problem Statement:]
        A robot who's location on a grid is denoted $R$ is located at the top-left corner of a $m$ x $n$ grid.
        The robot is trying to reach the finish square at the bottom right corner of the grid donoted as $F$.
        How many possible unique paths to $F$ exist starting at $R$, given the constraint that the robot can only move right one square, or down one square at any given point in time?
        
    \item[Input:]
        Two integers $m$ and $n$, which are the dimensions of the grid.
        
    \item[Output:] 
        An integer $unique\_paths(m,n)$.
        
    \item[Example:] For:\\
        $m = 3$, $n = 7$\\
        $unique\_paths(m,n) = 28$
        
    \item[Explanation:]
        There are 28 unique ways the robot can reach $F$ from $R$.
        Example Path:

        \begin{table}[htbp]
            \centering
            \begin{tabular}{|c|c|c|c|c|c|c|}
                \hline
                \textbf{R} & $>$ & V &  &  &  &  \\
                \hline
                 &  & $>$ & $>$ & $>$ & $>$ & V \\
                \hline
                 &  &  &  &  &  & \textbf{F} \\
                \hline
            \end{tabular}
        \end{table}

        
\end{description}

\subsection{Tabulation Approach to Unique Paths}
Notice that wherever the robot lands on the grid, it is possible to reach the finish square,
so we do not have to worry about cases where the robot cannot reach the finish square (backtracking).
Notice also that the number of paths from any square is the sum of the number paths to the right, and the number of paths if you go down.
We can use tabulation to solve the problem, by creating an $m$ x $n$ table called $dp$ where the value at $dp[i][j]$ represents the number of unique paths from that square to $F$.
The value at $dp[0][0]$ will hence contain the solution to the problem.
We can then initialize $dp[F]$ to 1, as there is one path from F to itself (the path contains zero moves).
The square directly above and to the left of $dp[F]$ can also be initialized to 1, as there is one path from them to $F$, going down or right respectively. This logic follows for all of the bottom row of the grid, and the last column of the grid.

\begin{table}[H]
    \centering
    \begin{tabular}{|c|c|c|c|c|c|c|}
        \hline
         &  &  &  &  &  & 1 \\
        \hline
         &  &  &  &  &  & 1 \\
        \hline
         & 1 & 1 & 1 & 1 & 1 & 1 \\
        \hline
    \end{tabular}
\end{table}

Now we simply fill the grid from the bottom right to the top left,
where the value of each square is the sum of the value below it and to its right.

\begin{table}[H]
    \centering
    \begin{tabular}{|c|c|c|c|c|c|c|}
        \hline
        28 & 21 & 15 & 10 & 6 & 3 & 1 \\
        \hline
        7 & 6 & 5 & 4 & 3 & 2 & 1 \\
        \hline
        1 & 1 & 1 & 1 & 1 & 1 & 1 \\
        \hline
    \end{tabular}
\end{table}

The top left square is where the robot was, so that is the number of paths from the robot to the finish square.

A sample Python implementation is given in \ref{fig:unique-paths}
\begin{figure}[H]
    \centering
    \begin{lstlisting}
    def unique_paths(m,n):
    
        # Initialize a m x n table of 1s
        dp = [[1] * n for _ in range(m)]
    
        # Fill in the table in a bottom-up manner
        for i in range(m-1,-1,-1):
            for j in range(n-1,-1,-1):
                # Leave last row and column as 1s
                if i == m-1 or j == n-1:
                    continue
                # Fill all other squares with the sum of the square below and to the right
                else:
                    dp[i][j] = dp[i+1][j] + dp[i][j+1]

        return dp[0][0]
    \end{lstlisting}
    \caption{Unique Paths Python Implementation}
    \label{fig:unique-paths}
\end{figure}

\subsection{Complexity Analysis of Unique Paths}\label{ca-unique-paths}

\begin{description}
    \item[Time Complexity:]
        The time complexity of filling an $m$ x $n$ table with values,
        where each value is calculated by two constant time lookups and a constant time addition, is $O(m * n)$.
        
    \item[Space Complexity:] 
        The space complexity of storing an $m$ x $n$ table where each field in the table contains a single integer is $O(m * n)$.

    \item[Overall:] Total:\\
        Time Complexity: $O(m * n)$\\
        Space Complexity: $O(m * n)$
    
\end{description}

\subsection{Optimization of the Unique Paths Problem}
Notice that for a given value in row $r$ in the table $dp$, $v$ can be calculated using just row $r$ and row $r-1$.
Since we compute the values in $dp$ rowwise from the bottom up, we do not actually need to store the entire $dp$ table in memory as the values which are two or more rows below the row we are computing at any given time do not contribute to our calculation.
Storing only two rows of the $dp$ table at a time reduces the space complexity from $O(m * n)$ to just $O(m)$.

A Python implementation of this optimization is given in Figure \ref{fig:unique-paths-optimized}.

\begin{figure}[H]
    \centering
    \begin{lstlisting}
    def unique_paths_optimized(m,n):
        # Initialize the bottom row of 1s
        oldRow = [1] * n
    
        # For each subsequent row in the grid
        for _ in range(m-1):
            # Create a new row which is initialized to all 1s
            newRow = [1] * n
            # Traverse the new row in reverse, without the last element
            for j in range(n - 2, -1, -1):
                # Each value in the new row is the sum of the value to the right and below
                newRow[j] = newRow[j+1] + oldRow[j]
            oldRow = newRow
    
        return oldRow[0]
    \end{lstlisting}
    \caption{Unique Paths Optimized Python Implementation}
    \label{fig:unique-paths-optimized}
\end{figure}

\subsection{Min Path Sum}
\begin{description}
    \item[Problem Statement:]
        Given a 2D array filled with non-negative numbers representing the 'cost' at that field in the input matrix,
        find the cost of the path from the top-left corner (denoted $R$) to the bottom-right corner (denoted $F$) which minimizes the sum of numbers along the path.
        (minimizes path cost).
        You can only move down or to the right at any point in time (no path can make negative progress).

    \item[Input:]
        A 2D array, called $cost$.
        
    \item[Output:]
        An integer $min_path_sum(cost)$, which represents the cost of the minimum path from $R$ to $F$.
        
    \item[Example:] For:\\
    $cost = $\\
    $[[1,3,1],$\\
    $[1,5,1],$\\
    $[4,2,1]]$\\

    $min_path_sum(cost) = 7$

    \item[Explanation:]
        Explanation: $1 + 3 + 1 + 1 + 1 = 7$ is the minimum path sum.
        \begin{table}[H]
            \centering
            \begin{tabular}{|c|c|c|}
                \hline
                \textbf{R} & $\rightarrow$ & $\downarrow$ \\
                \hline
                 &  & $\downarrow$ \\
                \hline
                 &  & \textbf{F} \\
                \hline
            \end{tabular}
        \end{table}
        
\end{description}

\subsection{Tabulation Approach to Minimum Path Sum}
We can solve this problem in a very similar manner to Unique Paths.
We start by creating a table $dp$ with the same dimensions as the input matrix, where $dp[i][j]$ represents the minimum path cost from $dp[i][j]$ to $F$.
We can initialize $F$ to be it's own cost in the input matrix (the cost of moving from $F$ to $F$ is $cost[F]$).

\begin{table}[H]
    \centering
    \begin{tabular}{|c|c|c|}
        \hline
         &  &  \\
        \hline
         &  &  \\
        \hline
         &  & 1 \\
        \hline
    \end{tabular}
\end{table}
Now we can iterate through the input matrix backwards using the same logic as in Unique Paths, but this time:


\begin{enumerate}
    \item If we are on the last column: $dp[i][j] = dp[i][j+1] + cost[i][j]$.
    [This is because there is no more squares to go right, so we must go down and incur that cost]

    \item If we are on the last row: $dp[i][j] = dp[i+1][j] + cost[i][j]$.
    [This is because there is no more squares to go down, so we must go right and incur that cost]

    \item Otherwise, $dp[i][j] = cost[i][j] +$ the minimum cost of going down, or going right.

\end{enumerate}

The top left of the grid will give us the minimum cost of reaching the bottom right position.

\begin{table}[H]
    \centering
    \begin{tabular}{|c|c|c|}
        \hline
        7 & 6 & 3 \\
        \hline
        8 & 7 & 2 \\
        \hline
        7 & 3 & 1 \\
        \hline
    \end{tabular}
\end{table}

A sample Python Implementation of this is given in Figure \ref{fig:min-path-sum}

\begin{figure}[H]
    \centering
    \begin{lstlisting}
    def min_path_sum(cost):
        rows, cols = len(cost), len(cost[0])
        
        # Initialize a table to store minimum path sums
        dp = [[0] * cols for _ in range(rows)]
    
        # Fill in the table in a bottom-up manner
        for i in range(rows-1,-1,-1):
            for j in range(cols-1,-1,-1):
                # Initialize the bottom right square to its own cost
                if i == rows-1 and j == cols-1:
                    dp[i][j] = cost[i][j]
    
                # When we are on the last row or col
                elif i == rows-1:
                    dp[i][j] = dp[i][j+1] + cost[i][j]
                elif j == cols-1:
                    dp[i][j] = dp[i+1][j] + cost[i][j]
    
                # Otherwise, choose the minimum cost path and add its cost
                # to the cost of this square.
                else:
                    dp[i][j] = cost[i][j] + min(dp[i+1][j], dp[i][j+1])
    
        return dp[0][0]
    \end{lstlisting}
    \caption{Min Path Sum Python Implementation}
    \label{fig:min-path-sum}
\end{figure}

\subsection{Complexity Analysis of Min Path Sum}
Let $m$ be the number of rows in $cost$ (the input matrix), and $n$ be the number of columns in $cost$.

\begin{description}
    \item[Time Complexity:]
        The time complexity of filling an $m$ x $n$ table with values,
        where each value is calculated by two constant time lookups and a constant time addition, is $O(m * n)$.
        
    \item[Space Complexity:] 
        The space complexity of storing an $m$ x $n$ table where each field in the table contains a single integer is $O(m * n)$.
        
    \item[Overall:] Total:\\
        Time Complexity: $O(m * n)$\\
        Space Complexity: $O(m * n)$
    
\end{description}

\subsection{Optimization of Minimum Path Sum}
We can use the exact same optimization as we did with Unique Paths, as each value in row $r$ can be calculated from just the values in row $r$ and row $r-1$.
Storing only two rows at a given time will reduce the Space complexity to $O(m)$.

A sample Python implementation is given in Figure \ref{fig:min-path-sum-optimized}
\begin{figure}[H]
    \centering
    \begin{lstlisting}
    def min_path_sum_optimized(cost):
        rows, cols = len(cost), len(cost[0])
        
        # Initialize the bottom row of 0s
        oldRow = [0] * cols
    
        # For each subsequent row in the grid
        for i in range(rows-1, -1, -1):
            # Create a new row which is initialized to all 0s
            newRow = [0] * cols
            # Traverse the new row in reverse, same logic as before.
            for j in range(cols - 1, -1, -1):
                if i == rows-1 and j == cols-1:
                    newRow[j] = cost[i][j]
                elif i == rows-1:
                    newRow[j] = newRow[j+1] + cost[i][j]
                elif j == cols-1:
                    newRow[j] = oldRow[j] + cost[i][j]
                else:
                    newRow[j] = cost[i][j] + min(oldRow[j], newRow[j+1])
            oldRow = newRow
    
        return oldRow[0]
    \end{lstlisting}
    \caption{Min Path Sum Optimized Python Implementation}
    \label{fig:min-path-sum-optimized}
\end{figure}
\subsection{Note on Optimization}
This two-row framework can be used to solve any pathing problem which is constrained such that negative progress is not possible.
This optimization can be taken even further. We have solved both problems rowwise starting from the bottom. This problem can equally be solved columnwise,
using the same optimization except this time storing two columns instead of two rows.
If we solve these problems rowwise in the case of $m \leq n$, and columnwise if $m > n$, we actually reduce the space complexity even further to $O(min(m,n))$.

\subsection{Extending Unique Paths and Min Path Sum to 3 Dimensions}
So far we have seen that Unique Paths and Min Path Sum can be solved for 2D array inputs.
These problems can also be trivially solved for 1D array inputs, and single scalar inputs.
But what if the input to the problem is not a 2D array but a 3D array? (But the constraint which prevents negative progress is still in place).

\subsection{Tabulation Approach to 3D-Unique Paths}
Consider an $m$ x $n$ x $k$ array called $input$, where we start at the top left at index $(0,0,0)$ (denoted $R$), and our target is in the bottom right at index $(m-1, n-1, k-1)$ (denoted $F$).
We can only move down at the current depth, right at the current depth, or "in" (meaning we increase the depth we are at).
The problem, like before, is to find the number of distinct paths from $R$ to $F$.
We can use a similar approach as with the 2D problem, but this time dp will need to be a 3D array to store the intermediate values.
Instead of summing the value to the right and below to get our current value, we sum the value to the right, below, and at the next depth level.
The value at index $dp[i][j][t]$ represents the number of unique paths from $input[i][j][t]$ to $input[m-1][n-1][k-1]$.
We will still initialize $dp[m-1][n-1][k-1]$ to 1.
To calculate the value at $dp[i][j][t]$, we will sum up $dp[i+1][j][t]$, $dp[i][j+1][t]$ and $dp[i][j][t+1]$, unless the value is at a border,
in which case we we will exclude that border from the sum, because going further in that direction is not possible\footnote{eg, if $i+1 = m$, we exclude $dp[i+1][j][t]$ from the sum as $dp[i+1][j][t]$ is out of bounds.}.
The value at $dp[0][0][0]$ will be our solution.

A sample python implementation is shown in Figure \ref{fig:3d-unique-paths}.

\begin{figure}[H]
    \centering
    \begin{lstlisting}
    def unique_paths_3d(m, n, k):
        dp = [[[0 for _ in range(k)] for _ in range(n)] for _ in range(m)]

        dp[m - 1][n - 1][k - 1] = 1

        for i in range(m - 1, -1, -1):
            for j in range(n - 1, -1, -1):
                for t in range(k - 1, -1, -1):
                    if i + 1 < m:
                        dp[i][j][t] += dp[i + 1][j][t]
                    if j + 1 < n:
                        dp[i][j][t] += dp[i][j + 1][t]
                    if t + 1 < k:
                        dp[i][j][t] += dp[i][j][t + 1]

        return dp[0][0][0]
    \end{lstlisting}
    \caption{3D Unique Paths Python Implementation}
    \label{fig:3d-unique-paths}
\end{figure}
\subsection{Complexity Analysis of 3D-Unique-Paths}

\begin{description}
    \item[Time Complexity:]
        We have proved that the time complexity of 2D Unique Paths is $O(m * n)$ in Section \ref{ca-unique-paths}.
        Similarly, we can see that the time complexity of 3D Unique Paths is $O(m * n * k)$.
        
        
    \item[Space Complexity:] 
        We have proved that the space complexity of 2D Unique Paths is $O(m * n)$ in Section \ref{ca-unique-paths}.
        Similarly, we can see that the space complexity of 3D Unique Paths is $O(m * n * k)$.
            
        
    \item[Overall:] Total:\\
        Time Complexity: $O(m * n * k)$\\
        Space Complexity: $O(m * n * k)$
\end{description}

\subsection{Note on this Approach}
We can trivially convert this 3D Unique Paths algorithm into a 3D Min Path Sum solution, and in fact we can convert any 2D problem with the constraint that negative progress is not allowed into a 3D problem using this approach.





\subsection{A Framework for Solving N-Dimensional Pathfinding Problems}
Consider a K-Dimensional pathfinding problem where no negative progress is allowed (similar to Unique Paths or Min Path Sum).
We can extend it to a K+1-Dimensional problem by simply considering a single new 'direction' in our calculation.
We have implemented this for Unique Paths in Figure \ref{fig:3d-unique-paths}.

We have shown that we can theoretically transform a K-dimensional pathfinding problem into a K+1-Dimensional pathfinding problem.
We can repeat this transformation N times in order to arrive at any N-Dimensional pathfinding problem.



\section{The Interstellar Problem Implementation Details}
Now that we have proved that The Interstellar Problem is solvable, we may attempt to implement a solution in Python.
This implemetation is tricky as we must simulate $n$ for loops where $n$ is the dimension count. The transformation from K to K+1-Dimensional pathfinding problems is trivial in theory, but requires thinking outside the box to implement in practice.
Numpy was used to 
access, store data in, and manipulate the input tensor as well as the $dp$ table.
This choice was made because Numpy allows us to access elements in n-rank tensors by providing a tuple of length $n$ as an index, rather than manually accessing each cell through a series of nested indices.
Numpy also allows us to easily initialialize n-rank tensors through the use of its $np.zeros(shape)$ function, rather than having to use something similar to Figure \ref{fig:zeros}:

\begin{figure}[H]
    \centering
    \begin{lstlisting}
    def initialize_array(dimensions):
        if len(dimensions) == 1:
            return [0] * dimensions[0]
        else:
            return [initialize_array(dimensions[1:]) for _ in range(dimensions[0])]
    \end{lstlisting}
    \caption{Example code to generate an n-rank tensor of zeros without Numpy}
    \label{fig:zeros}
\end{figure}

Numpy's $np.ndindex()$ is a function that provides an iterator yielding tuples of indices for a given shape.
It is particularly useful for iterating over the indices of n-dimensional arrays.
This function returns an object that can be iterated over, generating all possible index tuples for a specified shape.
\newpage
The pseudocode for The Interstellar Problem I is as follows:

\begin{enumerate}
    \item Initialize an n-dimensional array of zeros called $dp$ to store the number of paths for each cell, with $dp[R]$ set to 1.
    \item The $np.ndindex(shape)$ generates an iterator over all possible indices in the n-dimensional array.
    \item For each index, represented by the $current\_cell$:
    \item Iterate over each dimension using for $i$ in $range(num\_dimensions)$.
    \item Check if the current cell's index in the current dimension ($current\_cell[i]$) is greater than 0. If true, it means there is a valid cell to move from in that dimension.
    \item Create a copy of the current cell (denoted $prev\_cell$) and decrement the index in the current dimension ($prev\_cell[i] -= 1$). This represents the cell from which we are coming.
    \item Add the number of paths from the previous cell to the current cell in the $dp$ array ($dp[index] += dp[tuple(prev\_cell)]$).

\end{enumerate}

\section{Python Implementation of The Interstellar Problem 1}

Figure \ref{fig:nd-unique-paths} shows a Python Implementation of The Interstellar Problem I using the pseudocode described above.
The code is heavily commented so we can see the framework in action. Note that this approach can be trivially converted to calculate the Min Path Sum (The Interstellar Problem II) instead, or any other pathfinding problem through an N-dimensional field where negative progress is not allowed.

\begin{figure}[H]
    \centering
    \begin{lstlisting}
    import numpy as np

    def interstellar_1(dimensions):
        # Determine the number of dimensions
        num_dimensions = len(dimensions)
    
        # Initialize an n-dimensional array to store the number of paths for each cell
        shape = tuple(dimensions)
        dp = np.zeros(shape, dtype=int)
    
        # Set the number of paths for the starting cell to 1
        dp[(0,) * num_dimensions] = 1
    
        # Calculate the number of paths for each cell in the matrix
        for index in np.ndindex(shape):
            # Get the current index as an array
            current_cell = np.array(index)
            # Iterate over the array
            for i in range(num_dimensions):
                # If the current cell is not on the edge
                if current_cell[i] > 0:
                    # Add the previous cell's paths to the current cell [this happens from each valid direction in each dimension]
                    prev_cell = current_cell.copy()
                    prev_cell[i] -= 1
                    dp[index] += dp[tuple(prev_cell)]
    
        # The result is stored in the last cell of the matrix
        return dp[tuple(np.array(dimensions) - 1)]
    
    # Example usage:
    dimensions = (2,7,5)
    print(interstellar_1(dimensions))
    \end{lstlisting}
    \caption{Python Implementation of The Interstellar Problem I}
    \label{fig:nd-unique-paths}
\end{figure}

\subsection{Complexity Analysis of The Interstellar Problem I}
Let $n$ be the length of the tuple $dimensions$.
\begin{description}
    \item[Time Complexity:]
        The time complexity of filling a tensor of shape $dimensions$ with values,
        where each value is obtained through $n$ constant time additions is:\\$O((\prod_{i=1}^{n} dimensions_i) * n)$. This is because there are $\prod_{i=1}^{n} dimensions_i$ cells to fill, and each takes $n$ operations to complete, one addition from each dimension.
        
    \item[Space Complexity:] 
    The space complexity of filling a tensor of shape $dimensions$ with values,
    is $O(\prod_{i=1}^{n} dimensions_i)$.
    This is because there are $\prod_{i=1}^{n} dimensions_i$ cells to fill and store.
    
    \item[Overall:] Total:\\
        Time Complexity: $O((\prod_{i=1}^{n} dimensions_i) * n)$\\
        Space Complexity: $O(\prod_{i=1}^{n} dimensions_i)$

\end{description}


