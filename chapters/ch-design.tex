\chapter{Design}

\section{Introduction to Dynamic Programming}
Dynamic Programming has many definitions, but can be summarized as method of breaking down a larger problem into sub-problems, so that if you work through the sub-problems in the right order, building each answer on the previous one, you eventually solve the larger problem.
The two attributes a problem needs to have in order to be classified as a dynamic programming problem are as follows:

\begin{definition}[Optimal Substructure]
    A problem is said to have optimal substructure if an optimal solution to the problem can be deduced from optimal solutions of some or all of its subproblems.
\end{definition}

\begin{definition}[Overlapping Subproblems]
    A problem is said to have overlapping subproblems if the problem can be broken down into subproblems which can be reused several times or a recursive algorithm would solve the same subproblem more than once resulting in repeated work. (If the subproblems do not overlap, the algorithm is categorized as a "divide and conquer" algorithm rather than a dynamic programming algorithm.)
\end{definition} 
Once we have deduced that a problem has both of these properties, we can use dynamic programming principles in order to solve the problem in an efficient manner.
When solving a dynamic programming problem, it is common to start by implementing a brute force solution which explores all subproblems and returns a solution.
We can then extend our solution to use a cache to store the results of any subproblems encountered, such that when the subproblem is encountered again we do not need to re-compute the result, instead we can simply look up the cache in constant time.
This is known as "memoization", or "top-down dynamic programming".
We can then look for any patterns in the cache table which, given an initialization (usually the base case of the recursive solution), would allow us to compute the values stored in the cache without ever traversing the decision tree of the problem itself.
This is known as "tabulation", or "bottom-up dynamic programming".
In order to demonstrate this, we will use a simple problem called The Fibonacci Problem.

\subsection{Demonstration of Dynamic Programming Principles using the Fibonacci Problem}
\begin{description}
    \item[Problem Statement:]
        Compute the n'th number in the Fibonacci sequence.\footnote{Where $fib(1) = 1$, $fib(2) = 1$ and $fib(n) = fib(n-1) + fib(n-2)$}
        
    \item[Input:] 
        A positive integer $n$.
        
    \item[Output:] 
        An positive integer $fib(n)$.
        
    \item[Example:]
        For: 

        $n = 7$
        
        $fib(n) = 13$

    \item[Explanation:]
        The Fibonacci sequence is as follows: 1,1,2,3,5,8,13,...\\
        We can see that the 7th number in the sequence is 13.

\end{description}


\subsection{Brute Force Approach to The Fibonacci Problem}

We start with a brute force approach which will use recursion. 
The base cases are $fib(1) = 1$ and $fib(2) = 1$.
By definition, the recursive case is $fib(n) = fib(n-1) + fib(n-2)$.
An implementation of the brute force solution is given below.
A sample python implementation is shown in Figure \ref{fig:fibonacci-bf}.

\begin{figure}[H]
    \centering
    \begin{lstlisting}
    def fib_bf(n):
        if n <=2: return 1
        return fib_bf(n-1) + fib_bf(n-2)
    \end{lstlisting}
    \caption{Fibonacci Brute Force Python Implementation}
    \label{fig:fibonacci-bf}
\end{figure}

In order to understand just how inefficent this approach is, consider the calculation of fib(20).
The brute force approach will split this calculation into the calculation of fib(19) + fib(18).
Now, to calculate fib(19), we split it into fib(18) + fib(17), and to calculate fib(18) we split it into fib(17) + fib(16).
Since the original problem was to calculate fib(19) + fib(18), and we need fib(18) to calculate fib(19), the calculation of fib(18) is repeated. 


\subsection{Complexity Analysis of the Brute Force Approach to the Fibonacci Problem}

\begin{description}
    \item[Time Complexity:]
    At each step in the calculation of $fib(n)$, we make two 'branches', where one calculates $fib(n-1)$ and the other calculates $fib(n-2)$.
    This branching factor leads to an exponential growth in the number of function calls.
    The number of function calls grows exponentially with $n$, as each level of the tree doubles the number of function calls.
    Therefore the time complexity of this approach is $2 + 2^2 + 2^3 + ... + 2^n$ which is $O(2^n)$.
        
    \item[Space Complexity:] 
        In the brute force approach to compute Fibonacci numbers, the space complexity is influenced by the recursive calls, each of which adds a frame to the call stack. However, as the recursion progresses, some of these frames can be discarded once their corresponding Fibonacci values have been computed.
        Specifically, at any point during the recursion, we only need to keep track of the previous two Fibonacci numbers. Therefore, the maximum depth of the call stack at any point is at most $n$ due to the recursion.
        This means that the space complexity of the brute force approach to compute Fibonacci numbers is $O(n)$.
        
        
    \item[Overall:] Total:\\
        Time Complexity: $O(2^n)$\\
        Space Complexity: $O(n)$
        
\end{description}

\subsection{Memoization Approach to The Fibonacci Problem}
Because we have optimal substructure\footnote{The optimal solution to fib(n-1) + the optimal solution to fib(n-2) will always be the optimal solution to fib(n).}
and overlapping subproblems\footnote{The calculation of fib(n-1) contains the calculation of fib(n-2).}, we can make this calculation more efficient through the use of memoization.
This simple adjustment involves storing a $(key, value)$ table called $memo$, where the key is an intermediate subproblem and the value is the intermediate result of that subproblem.
Now, for any subproblem, we first check if the result is in memo and if it is, we return the result of that calculation in constant time.
If the subproblem is not in $memo$, we calculate the intermediate result and cache it in $memo$.
A sample python implementation is shown in Figure \ref{fig:fibonacci-memo}.

\begin{figure}[H]
    \centering
    \begin{lstlisting}
    def fib_memo(n, memo={}):
        if n <= 2:
            return 1
        if n in memo:
            return memo[n]
        memo[n] = fib_memo(n-1, memo) + fib_memo(n-2, memo)
        return memo[n]
    \end{lstlisting}
    \caption{Fibonacci Memoization Python Implementation}
    \label{fig:fibonacci-memo}
\end{figure}

\subsection{Complexity Analysis of the Memoization Approach to The Fibonacci Problem}

\begin{description}
    \item[Time Complexity:]
        Since each subproblem is only ever computed once, and any repeated subproblems are handled with a constant time table lookup,
        the time complexity depends only on the amount of subproblems.
        Since there are $n$ possible subproblems for any given input $n$,
        the time complexity is reduced to $O(n)$

    \item[Space Complexity:] 
        The space complexity remains determined by the recursion call stack, at $O(n)$.
        We also have to store the memo table, which contains an integer solution to each of the $n$ subproblems.
        This is also $O(n)$, giving us a total $O(2n)$ space complexity.
        This can be simplified to $O(n)$.

    \item[Overall:] Total:\\
        Time Complexity: $O(n)$\\
        Space Complexity: $O(n)$
        
\end{description}

\subsection{Tabulation Approach to the Fibonacci Problem}
    
With the memoization approach, we saw how to compute the solution top-down, starting at $fib(n-1) + fib(n-2)$,
arriving at the base cases, and working up from there.
Notice that this step is unnecessary.
If we can deduce $fib(3)$ from $fib(2) + fib(1)$ (both of which are given in the base case),
and $fib(4)$ from $fib(3)$ and $fib(2)$,
we can work bottom-up until we arrive at $fib(n)$.
This reduces the space complexity from $O(2n)$ to $O(n)$,
as all we need to do is store the table.
It is common practice to refer to the table as $dp$ in tabulation approaches.

A sample python implementation is shown in Figure \ref{fig:fibonacci-dp}.

\begin{figure}[H]
    \centering
    \begin{lstlisting}
    def fib_dp(n):
        if n <= 2:
            return 1
    
        dp = [0] * (n + 1)
        dp[1] = 1
    
        for i in range(2, n + 1):
            dp[i] = dp[i - 1] + dp[i - 2]
    
        return dp[n]
    \end{lstlisting}
    \caption{Fibonacci Tabulation Python Implementation}
    \label{fig:fibonacci-dp}
\end{figure}


\section*{Space Optimized Approach to the Fibonacci Problem}
We can often save space with the tabulation approach by releasing parts of the dp table which are not in use from memory.
In this case, notice that we only need $fib(n-1) and fib(n-2)$ to deduce the result of $fib(n)$.
The rest of the table does not need to be stored. 
We can achieve this by storing just two variables, $prev$ and $curr$.
For an arbitrary value $k$, $curr$ represents the value of $fib(k)$, $prev$ represents the value of $fib(k-1)$.
We can calculate the result of $fib(n+1)$ from $prev$ and $curr$, then update $curr$ to the result, and $prev$ to what $curr$ was.
Starting at $curr=1$ and $prev=0$ and repeating this $n-1$ times will make $curr = fib(n)$.

A sample python implementation is shown in Figure \ref{fig:fibonacci-optimized}.
\begin{figure}[H]
    \centering
    \begin{lstlisting}
    def fib_optimized(n):
        if n <= 1:
            return n
        
        prev, curr = 0, 1
        for _ in range(n-1):
            prev, curr = curr, prev + curr
            
        return curr
    \end{lstlisting}
    \caption{Fibonacci Optimized Python Implementation}
    \label{fig:fibonacci-optimized}
\end{figure}

\subsection{Complexity Analysis of the Tabulation Approach to the Fibonacci Problem}

\begin{description}
    \item[Time Complexity:]
        The time complexity remains unchanged at $O(n)$.

    \item[Space Complexity:] 
        Since we are only storing two variables of constant size at a time,
        and there is no recursion, the space complexity of this optimized version is $O(1)$.

    \item[Overall:] Total:\\
        Time Complexity: $O(n)$\\
        Space Complexity: $O(1)$
        
\end{description}



\section{Dynamic Programming Summary}

In summary, the "dynamic programming way of thinking" involves:
\begin{enumerate}
    \item Creating a brute force solution.
    \item Figuring out if the optimal substructure property holds.
    \item Identifying the repeating and overlapping subproblems.
    \item Introducing memoization to the brute force solution to eliminate repeated work.
    \item Using tabulation to try to deduce the memoization table bottom-up rather than top-down.
    \item Looking for ways to optimize space in the tabulation approach by reducing the size of the table.
\end{enumerate}

Using the Fibonacci example, we have demonstrated the way of thinking about a problem which is dynamic programming.
We have went from an $O(2^n)$ time and $O(n)$ space complexity recursive solution to an $O(n)$ time and $O(1)$ space complexity solution using dynamic programming principles.
The Project Notebook contains an in depth analysis of 8 well known dynamic programming problems, followed by my own original dynamic programming problems and solutions.
Where applicable, the problems analyzed contain:

\begin{enumerate}
    \item The problem statement, and a deep explanation of the problem with examples. This may contain example greedy algorithms and proofs of why they do not actually work for the given problem.
    \item A comprehensive brute force algorithm, with an implementation and complexity analysis.
    \item A short informal proof of why the optimal substructure and overlapping subproblems attributes hold.
    \item An explanation of how memoization is used in the problem, with an implementation and complexity analysis.
    \item An explanation of how tabulation is used in the problem, with an implementation and complexity analysis.
    \item An explanation of how we can space optimize the tabulation solution, with an implementation and complexity analysis.

\end{enumerate}

In the Project Notebook, the tabulation approach of each of the problems comes with a $printTable$ flag which, when set to $True$, displays the table which has been calculated for the specific problem.

\section{The Coin Change Problem}
\begin{description}
    \item[Problem Statement:]
        Given a list of denominations of coins $D$ and an integer amount $a$, compute the minimum amount of coins (where each coin's denomination $\in D$) needed to sum exactly to the given amount $a$.
        
    \item[Input:] 
        An integer array $D$ of possible coin denominations, and an integer amount $a$.
        
    \item[Output:] 
        An integer $r$, which represents the minimum amount of coins with denominations $\in D$ needed in order to sum exactly to $a$. If this cannot be done, return $-1$.
        
    \item[Example:]
        For: 

        $D = [1, 5, 10, 20]$

        $a = 115$

        $r = 7$

    \item[Explanation:]
        The minimum amount of coins with denominations in $D$ needed to sum to $a$ is 7.

        These coins are: $[20,20,20,20,20,10,5]$

\end{description}

\subsection{Greedy Approach to the Coin Change Problem}

Algorithm~\ref{algo:coin-change-greedy} shows a greedy approach to the coin change problem.

\begin{algorithm}[H]
    \caption{Greedy Approach to the Coin Change Problem}
    \label{algo:coin-change-greedy}
    \KwIn{List of denominations of coins $D$ and an amount $a$}
    \KwOut{$r$, The minimum number of coins required to make change for $a$}
    Sort $D$ in ascending order\;
    $r \leftarrow 0$\;
    $total \leftarrow 0$\;
    \While{$\text{total} < a$}{
        \If{$|D| = 0$}{
            \KwRet{-1}\;
        }
        \If{$\text{total} + D[-1] > a$}{
            $D.\text{pop()}$\;
        }
        \Else{
            $\text{total} \leftarrow \text{total} + D[-1]$\;
            $r \leftarrow r + 1$\;
        }
    }
    \KwRet{$r$}
\end{algorithm}

In Algorithm~\ref{algo:coin-change-greedy}, we choose the coin with the largest value which will not make the total exceed a.

\subsection{Optimality of the Greedy Approach to the Coin Change Problem}

This algorithm is not optimal, and we can prove this by counter-example.
Take:$$D = [5,4,3,2,1], a = 7$$Given these inputs, the greedy result is: $r1 = 3$  ([5,1,1]).\\
The optimal solution for these inputs is: $r2 = 2$ ([4,3]).\\
We see that $r1 > r2$, meaning the greedy approach does not find the miminized solution.

\subsection{Correctness of the Greedy Approach to the Coin Change Problem}

The algorithm is also not correct, and we can prove this by another counter-example.
Take:$$D = [4,3],a = 6$$Given these inputs, the greedy result is: $r1 = -1$  ([4]).\\
The optimal solution for these inputs is: $r2 = 2$ ([3,3]).\\
We see that the greedy approach fails when a solution is indeed possible, as shown by $r2$.\\
Since we have shown that the greedy approach is neither correct nor optimal, we move on to the brute force solution.\\


\subsection{Brute Force Approach to the Coin Change Problem}

The brute force approach to the coin change problem involves generating all possible coin combinations, and checking if any of them sum exactly to $a$.
Of the ones that do, we return the minimum length.
To try all possible coin combinations, we can subtract each coin denomination $c \in D$ from $a$, as long as $a - c >= 0$.
We can repeat this step for each result obtained from this calculation (replacing $a$ with the intermediate result), until all possible coin combinations are explored.
We can keep track of the shortest path through the resulting tree which has a leaf value of 0, to avoid storing the entire tree in memory.
We return the length of the shortest path as $r$.\\


% TODO NOTE: PLOT OF ALGORITHM HERE

A sample python implementation is shown in Figure \ref{fig:coin-change-bf}.

\begin{figure}[h]
    \centering
    \begin{lstlisting}
    def coin_change_bf(D, a):
        def dfs(a):
            if a == 0:
                return 0
            if a<0:
                return float('inf')
            return min([1+dfs(a-c) for c in D])
        minimum = dfs(a)
        return minimum if minimum < float("inf") else -1
    \end{lstlisting}
    \caption{Coin Change Brute Force Python Implementation}
    \label{fig:coin-change-bf}
\end{figure}

\subsection{Complexity Analysis of the Brute Force Approach to the Coin Change Problem}

\begin{description}
    \item[Time Complexity:]
    For the worst case scenario, let's assume each coin denomination $c \in D < a$ such that each node which is not a leaf node has $|D|$ children. This means we have $|D|$ recursive calls at the first level, $|D|^{2}$ at the second level, $|D|^{n}$ at the $n$'th level.

    The total number of recursive calls in this scenario is $|D| + |D|^{2} + ... + |D|^{a}$ which is $O(|D|^a)$.
    
    Therefore the time complexity is $O(|D|^{a})$. This is because at each step, there are $|D|$ choices (coin denominations) to consider, and the recursion depth is at most $a$.
    
        
    \item[Space Complexity:] 
        We do not store the entire tree in memory, only the current path.

        The space complexity is determined by the maximum depth of the recursion stack. In the worst case, the recursion depth is equal to the target amount $a$. Therefore, the space complexity is $O(a)$.
        
        
    \item[Overall:] Total:\\
        Time Complexity: $O(|D|^a)$

        Space Complexity: $O(a)$
        
\end{description}

\subsection{Memoization Approach to the Coin Change Problem}

We can trivially see that the problem has the optimal substructure property.
In the brute force algorithm, we have a chance to arrive at a value multiple times.
This is because an intermediate value can be made up of different combinations of coins (eg, 3 can be made up of (2,1) or (1,1,1)).
This demonstrates the overlapping subproblems property.
Therefore, we can use memoization to prevent repeated calculations of the optimal number of coins needed to make up a given sub-amount.\\
For every path in the search tree, we can store intermediate results in a table, so that the next time we arrive at a value, eg. 3, we don't have to repeat the work in finding the minimum amount of extra coins needed to sum to a.
Instead we can simply look in the table with a constant time lookup.
This optimization reduces search time greatly, as seen in subsection \ref{subsec:ca-coin-change-memo}.\\
A sample python implementation is shown in figure \ref{fig:coin-change-memo}.

\begin{figure}[H]
    \centering
    \begin{lstlisting}
    def coin_change_memo(D, a):
    memo = {}
    def dfs(a):
        if a == 0:
            return 0
        if a < 0:
            return float('inf')
        if a in memo:
            return memo[a]
        
        memo[a] = min([1+dfs(a-c) for c in D])
        return memo[a]
            
    res = dfs(a)
    return res if res < float("inf") else -1
    \end{lstlisting}
    \caption{Coin Change Memoization Python Implementation}
    \label{fig:coin-change-memo}
\end{figure}

\subsection{Complexity Analysis of the Memoization Approach to the Coin Change Problem}\label{subsec:ca-coin-change-memo}

\begin{description}
    \item[Time Complexity:]
    %   CITATION HERE TO THE PYTHON MANUAL
        Each unique subproblem is evaluated once, and the next time it is encountered it is retrieved from the memoization table with a constant time lookup\footnote{Python dictionary lookups have an expected $O(1)*$ time complexity.}.
        As there are $|D| * a$ unique subproblems in the worst case\footnote{$|D|$ constant time subtractions from any intermediate value $v$ where $0 \leq v \leq a$.}, the time complexity to solve all of them is $O(|D| * a)$.
    
        
    \item[Space Complexity:] 
        We need to store the $memo$ table in memory.
        The memoization table is represented by a lookup data structure where the keys range from 0 to a,
        representing the solution to each unique subproblem.
        Hence, the memory required to store the table is of order $O(a)$.
        
    \item[Overall:] Total:\\
        Time Complexity: $O(|D| * a)$

        Space Complexity: $O(a)$
        
\end{description}

\subsection{Tabulation Approach to the Coin Change Problem}

Instead of doing a dfs to fill in the memo table, we can calculate the values in the memo table directly, and extract the answer from there.
We will call the $memo$ table $dp$, as we are no longer doing memoization, but tabulation.
$dp[i]$ represents the minimum amount of coins needed to get the amount $i$.
Consider the example:$$D = [5,4,3,1], a = 7$$
We initialize each $dp[i]$ to contain infinity.
We know that $dp[0] = 0$ as it takes $0$ coins to add up to an amount of $0$. We can initialize this in our table.
Now we can deduce $dp[1], dp[2], ... dp[a]$. $dp[a]$ will contain $r$.
To get $dp[i]$, we will look at each coin $c \in D$ in sequence.
For each $c \in D$, we take $i - c$ to get $t$, and look for $dp[t]$ if it exists.
Our intermediate result is $1 + dp[t]$.
If this result is less than the current $dp[i]$ and is not negative, we update $dp[i] \leftarrow 1 + dp[t]$.\\
The logic of this is that the amount of coins it takes to make the amount $dp[i]$ is the amount of coins it takes to make the amount $dp[t]$ plus one.
The logic is demonstrated with the examples:

Example 1: Calculating $dp[1]$
$$dp[0]=0$$
$$dp[1]=\infty$$
$$dp[2]=\infty$$
$$dp[3]=\infty$$
$$dp[4]=\infty$$
$$dp[5]=\infty$$
$$dp[6]=\infty$$
$$dp[7]=\infty$$
To calculate $dp[1] (i=1)$:\\
For $c \in D = [5,4,3,1]$\\
$t = i-c = -4$, ignore because negative.\\
$t = i-c = -3$, ignore because negative.\\
$t = i-c = -2$, ignore because negative.\\
$t = i-c = 0$\\
Look up the value of $dp[t] = 0$.\\
Now we take $1 + dp[0] = 1$.\\
This means a possible solution to $dp[1]$ is 1.\\
Since $1 < \infty$, we update $dp[1] \leftarrow 1$

Example 2: Calculating $dp[7]$
$$dp[0]=0$$
$$dp[1]=1$$
$$dp[2]=2$$
$$dp[3]=1$$
$$dp[4]=1$$
$$dp[5]=1$$
$$dp[6]=2$$
$$dp[7]=\infty$$
To calculate $dp[7] (i=7)$:\\
For each $c \in D = [5,4,3,1]$:\\
$t = i-c = 2$, $1 + dp[2] = 3$, $3 < \infty$, update $dp[7] \leftarrow 3$\\
$t = i-c = 3$, $1+dp[3] = 2$, $2 < 3$, update $dp[7] \leftarrow 2$\\
$t = i-c = 4$, $1+dp[4] = 2$, $2 = 2$, ignore.\\
$t = i-c = 6$, $1+dp[6] = 3$, $3 > 2$, ignore.\\
We conclude that the minimum solution to $dp[7]$ is 2, achieved by adding a 4 coin to $dp[3]$, which is achieved by adding a 3 coin to $dp[0]$\\
A sample python implementation is shown in Figure \ref{fig:coin-change-dp}.\\

\begin{figure}[H]
    \centering
    \begin{lstlisting}
    def coin_change_dp(D,a):
        dp=[float('inf')] * (a + 1)
        dp[0] = 0
    
        for i in range(1, a+1):
            for c in D:
                t = i - c
                if t >= 0:
                    dp[i] = min(dp[i], 1+dp[t])
    
        return dp[a] if dp[a] != float('inf') else -1
    \end{lstlisting}
    \caption{Coin Change Tabulation Python Implementation}
    \label{fig:coin-change-dp}
\end{figure}

\subsection{Complexity Analysis of the Tabulation Approach to the Coin Change Problem}

\begin{description}
    \item[Time Complexity:]
        % REFERENCE TO PYTHON MANUAL
        For the worst case scenario, we need to iterate for all $0 \leq i \leq a$.
        And for each $i$, we need to iterate over each coin $c \in D$.
        All other operations within the loops are constant time lookups and subtractions, so the time complexity is $O(|D| * a)$
            
    \item[Space Complexity:] 
        The space complexity is determined by the size of the $dp$ array. This array is always of size $a+1$.
        Therefore the space complexity is $O(a)$
        
    \item[Overall:] Total:\\
        Time Complexity: $O(|D| * a)$

        Space Complexity: $O(a)$
        
\end{description}

\section{Longest Increasing Subsequence}
\begin{verbatim}

# Longest Increasing Subsequence

Given an array nums, return the length of the longest strictly increasing subsequence. A subsequence does not have to be contiguous.

Example: nums = [2,5,3,7,101,18]

Output: 4

Explanation: The subsequence [2,5,7,101] is the longest increasing subsequence, with length 4.

Much like coin change, this problem appears trivial at first glance. One may attempt to be greedy as follows:

## Longest Increasing Subsequence Greedy Approach

Input: nums

r := 0

index := 0

cur := nums[0]

While index < |nums|:

    if nums[index] > cur:

        r += 1

        cur := nums[index]

    index +=1

Return r

In this greedy algorithm we iterate through nums keeping track of the current max value enountered, incrementing our result each time a new larger value is encountered.

### Optimality of the Greedy approach

This algorithm is not optimal however, and we can prove this by counter-example.

A counter example is nums = [10,9,2,5,3,7,101,18]

Given these inputs, the greedy result is: r1 = 2  ([10,101])

An optimal solution for these inputs is: r2 = 4 ([2,3,7,101,18])

We see that r1 > r2, meaning the greedy approach does not find the maximised solution.

We therefore need a more sophisticated approach.

## Longest Increasing Subsequence Brute Force

We can try a brute force approach, where we start at index 0, and for each index choose wether we should exclude it from the subsequence or include it in the subsequence. We keep track of the prev_index, which represents the last index we included in the result, and the current_index, which is the index for which we are making the choice.

This will generate all possible increasing subsequenes.

We keep track of the longest increasing subsequence length, and return it.

Below is an implementation of this algorithm.

def length_of_lis_bf(nums):
    def dfs(prev_index, current_index):
        # Base case: reached the end of the sequence
        if current_index == len(nums):
            return 0

        # Case 1: Exclude the current element
        exclude_current = dfs(prev_index, current_index + 1)

        # Case 2: Include the current element if it is greater than the previous one
        include_current = 0
        if prev_index < 0 or nums[current_index] > nums[prev_index]:
            include_current = 1 + dfs(current_index, current_index + 1)

        # Return the maximum length of the two cases
        return max(exclude_current, include_current)

    # Start the recursion with initial indices (-1 represents no previous index)
    return dfs(-1, 0)

print(length_of_lis_bf([10,9,2,5,3,7,101,18]))

## Longest Increasing Subsequence Brute Force Complexity Analysis

Let n be the length of nums.

Time Complexity:

For the worst case scenario, There are n indices to consider. There are two subtrees at each decision, one where we include the current index, and one where we do not.

This brings the time complexity to O(2^n)

Space Complexity:

The space complexity is determined by the recursion depth.

Therefore the space complexity is O(n)

Overall:

Time Complexity: O(2^n)

Space Complexity: O(n)

## Longest Increasing Subsequence Memoization

We can use memoization to avoid repeating subproblems, such as when we are deciding wether the next element should be added or not for multiple subsequences ending in the same element.

Before proceeding with the recursive calls, the function checks if the result for the current combination of prev_index and current_index is already computed and stored in the memo dictionary. If it is, the stored result is returned immediately. This optimization allows the algorithm to avoid repeating work, speeding up the runtime.

def length_of_lis_memo(nums):
    if not nums:
        return 0

    memo = {}  # Memoization dictionary to store computed results

    def dfs(prev_index, current_index):
        if current_index == len(nums):
            return 0

        if (prev_index, current_index) in memo:
            return memo[(prev_index, current_index)]
        
        exclude_current = dfs(prev_index, current_index + 1)

        include_current = 0
        if prev_index < 0 or nums[current_index] > nums[prev_index]:
            include_current = 1 + dfs(current_index, current_index + 1)

        
        # Save the result in the memoization dictionary
        memo[(prev_index, current_index)] = max(include_current, exclude_current)

        return memo[(prev_index, current_index)]

    return dfs(-1, 0)

print(length_of_lis_memo([10,9,2,5,3,7,101,18]))

## Longest Increasing Subsequence Memoization Complexity Analysis

Let n be the length of nums.

Time Complexity:

For each unique combination of (prev_index, current_index), the algorithm either calculates the result or looks it up in the memoization table. Since there are at most n choices for each index, the time complexity for a single subproblem is O(n).

The algorithm explores all combinations of prev_index and current_index. There are at most n choices for current_index and, in the worst case, n choices for prev_index for each current_index. Therefore, the total number of unique subproblems is O(n^2).

Space Complexity: 

The space complexity is increased to O(n^2), as the memo table needs to store all n^2 combinations in the worst case.

Overall:

Time Complexity: O(n^2)

Space Complexity: O(n^2)

## Longest Increasing Subsequence Tabulation

We can use tabulation to build a table from which we can deduce the result, similar to the coin change problem.

We know that starting at the last index will result in an increasing subsequence of length 1. We can work backwards, for the second last, third last ect.. deciding if including that element will result in a longer increasing subsequence or not, and storing the longest possible increasing subsequence starting at each index until we reach index zero.

We create a table called dp of size len(nums), where dp[i] represents the longest increasing subsequence starting at index i in nums.

Lets take the example nums = [1,2,4,3]

We initialize dp[3] to 1, as the longest increasing subsequence starting at index 3 is 1.

Consider nums[2] = 4

We can either take nums[2] by itself, or include nums[2] in any subsequence at any index that comes after it (if it maintains the property of an increasing subsequence). Including it would make dp[2] = 1+dp[3], Excluding it would make dp[2] = 1.

Since including it would not result in an increasing subsequence, we must exclude it, so dp[2] = 1

Now Conisder nums[1] = 2

We can either take it by itself or include it in any subsequence at any index that comes after it. Including it would make dp[1] = max(1+dp[2], 1+dp[3]), taking it by itself would make dp[1] = 1

We choose the option which maximizes the value of dp[1], which is 1+dp[2] (or equally 1+dp[3]) = 2

So for dp[i], by the same logic, we simply put max(1,1+dp[j1],1+dp[j2],1+dp[j3]...) (only include 1+dp[jx] in the max function if nums[i] < nums[jx], to maintain increasing subsequence property)

def length_of_lis_dp(nums, printTable = False):
    dp = [1] * len(nums)

    for i in range(len(nums)-1,-1,-1):
        for j in range(i+1,len(nums)):
            if nums[i] < nums[j]:
                dp[i] = max(dp[i], 1+dp[j])

    if printTable:
        print(dp)

    return max(dp)

print(length_of_lis_dp([10,9,2,5,3,7,101,18], printTable=True))

## Longest Increasing Subsequence Tabulation Complexity Analysis

Let n be the length of nums.

Time Complexity:

For the worst case scenario, we need to perform a double nested iteration over nums.

All other operations within the loops are constant time lookups and max(a,b), so the time complexity is O(n^2)

Space Complexity:

The space complexity is determined by the size of the dp array. This array is always of size n.

Therefore the space complexity is O(n)

Overall:

Time Complexity: O(n^2)

Space Complexity: O(n)


\end{verbatim}

\section{Max Subarray Sum}
\begin{description}
    \item[Problem Statement:]
        Given an integer array, return the largest value that a subarray of the array sums to. A subarray is a contiguous subsequence. This means all elements of the subsequence are strictly consecutive in the original sequence.

    \item[Input:]
        An integer array $nums$.
        
    \item[Output:]
        An integer $max\_sum$. 
        
    \item[Example:] For:\\
        $nums =  [-2,1,-3,4,-1,2,1,-5,4]$\\
        $max\_sum = 6$
        
    \item[Explanation:]
        $[4,-1,2,1]$ is the subarray of $nums$ that has the largest sum, 6.
        
\end{description}


\subsection{Brute Force Approach to Max Subarray Sum}
The brute force way to solve this problem is to generate every possible subarray of $nums$
starting with the subarray containing only $nums[0]$, and ending with the entirety of $nums$.
Then sum each subarray and get the maximum of these sums.
Instead of generating and storing every subarray, we can iterate over each subarray in place by iterating over $nums$ with a $start$ pointer marking the start of the subarray,
and iterate over the rest of the array with an $end$ pointer marking the end of the subarray. This reduces the space complexity from O(n) to O(1) (see section \ref{subsec:ca-max-subarray-sum-bf}).
We can improve this further by keeping track of a $current\_sum$ and $max\_sum$ and updating them dynamically as we execute the $end$ loop rather than summing each subarray after it is generated,
reducing the time complexity from $O(n^3)$ to $O(n^2)$ (see section \ref{subsec:ca-max-subarray-sum-bf}).

A sample Python implementation is shown in Figure \ref{fig:max-subarray-sum-bf}.

\begin{figure}[H]
    \centering
    \begin{lstlisting}
    def max_subarray_sum_bf(nums):
        if not nums:
            return 0
    
        n = len(nums)
        max_sum = float('-inf')
    
        for start in range(n):
            current_sum = 0
            for end in range(start, n):
                current_sum += nums[end]
                max_sum = max(max_sum, current_sum)
    
        return max_sum
    \end{lstlisting}
    \caption{Max Subarray Sum Brute Force Python Implementation.}
    \label{fig:max-subarray-sum-bf}
\end{figure}



\subsection{Complexity Analysis of the Brute Force Approach to MSS}\label{subsec:ca-max-subarray-sum-bf}
Let n be the length of nums.
\begin{description}
    \item[Time Complexity:]
        Due to the use of a double nested for loop,
        generating all subarrays of $nums$ has a time complexity of $O(n^2)$.
        All other operations such as adding to the $current\_sum$, resetting the $current\_sum$ and getting the max of $current\_sum$ and $max\_sum$ are constant time.
        If we were to sum each subarray individually rather than keeping a $current\_sum$, which would be an $O(n)$ operation, the total complexity would be increased to $O(n^3)$.

    \item[Space Complexity:] 
        All other variables stored such as $max\_sum$ and $current\_sum$ are constant space.
        The number of variables stored does not increase as $n$ increases, hence the space complexity is $O(1)$.
        If instead of iterating over each subarray in place, we stored the current subarray separately, the space complexity would be increased to $O(n)$.

    \item[Overall:] Total:\\
        Time Complexity: $O(n^2)$\\
        Space Complexity: $O(1)$
    
\end{description}

    
\subsection{Kadanes Algorithm for Max Subarray Sum}
    
There is a way to find the max subarray sum in a single iteration of $nums$.
This is done using the famous Kadane's algorithm.
Kadane's algorithm is shown in Algorithm \ref{algo:kadanes}: 

\begin{algorithm}[H]
    \caption{Kadane's Algorithm}
    \label{algo:kadanes}
    \KwIn{An integer array $nums$.}
    \KwOut{An integer $max\_sum$, the max subarray sum of $nums$.}
    $max\_sum \leftarrow 0$\;
    $current\_sum \leftarrow 0$\;
    \ForEach{$i \in nums$}{
        \If{$current\_sum < 0$}{
            $current\_sum \leftarrow 0$\;
        }
        $current\_sum += i$\;
        \If{$current\_sum > max\_sum$}{
            $max\_sum \leftarrow current\_sum$\;
        }
    }
    \KwRet{$max\_sum$}
\end{algorithm}
\subsection*{Explanation of Kadanes Algorithm}
In Kadane's algorithm, we iterate over $nums$, and at each step increment $current\_sum$ by the value of the current element. 
If $current\_sum$ becomes negative, $current\_sum$ is reset to zero. 
The $max\_sum$ is updated to $current\_sum$ whenever $current\_sum$ becomes greater than $max\_sum$.
If the array consists entirely of negative numbers, the algorithm will return 0 for the maximum subarray sum.
If you want to modify the algorithm such that empty subarrays are not allowed, you can initialize $max\_sum$ and $current\_sum$ to the first element of the array instead of 0.
This way, the algorithm will return the largest single negative element if the array consists entirely of negative numbers.
Kadane's algorithm is considered a dynamic programming algorithm even though it does not use a table, because it satisfies the necessary criteria:

\textbf{Optimal Substructure:} The problem of finding the maximum subarray can be divided into smaller subproblems, where the solution to the problem at each index depends on the solution to the problem at the previous index.

\textbf{Overlapping Subproblems:} The subproblems in Kadane's algorithm overlap, as the solution to the problem at each index relies on the solution to the problem at the previous index.

A sample Python implementation is shown in Figure \ref{fig:kadanes}.

\begin{figure}[H]
    \centering
    \begin{lstlisting}
    def max_subarray_sum_kadanes(nums):
        if not nums:
            return 0
    
        max_sum = current_sum = nums[0]
    
        for num in nums[1:]:
            if current_sum < 0:
                current_sum = 0
            current_sum += num
            max_sum = max(max_sum, current_sum)
    
        return max_sum
    \end{lstlisting}
    \caption{Kadane's Algorithm Python Implementation}
    \label{fig:kadanes}
\end{figure}



\subsection{Complexity Analysis of Kadane's Algorithm}
Let n be the length of nums.
\begin{description}
    \item[Time Complexity:]
        This algorithm consists of a single iteration of nums, which is $O(n)$.
        All other operations such as updating the $current\_sum$, $max\_sum$ are constant time.
        
    \item[Space Complexity:] 
        All variables stored such as $max\_sum$ and $current\_sum$ are constant space,
        and we do not store more information as $n$ grows. Hence the space complexity is $O(1)$.

        
    \item[Overall:] Total:\\
        Time Complexity: $O(n)$\\
        Space Complexity: $O(1)$
    
\end{description}
\newpage

\section{Longest Alternating Subsequence}
\begin{description}
    \item[Problem Statement:]
        Given an array, find the length of the longest alternating subsequence in the array\footnote{Subsequences do not have to be continuous.}.
        A sequence ${x1,x2,x3,x4...xn}$ is alternating if its elements satisfy one of the following:

        $x1>x2<x3>x4<...$
        
        $x1<x2>x3<x4>...$
        
    \item[Input:]
        An integer array $nums$.
        
    \item[Output:] 
        An integer $max\_length$.
        
    \item[Example:] For:\\
        $nums = [1,17,5,10,13,15,10,5,16,8]$\\
        $max_length = 7$

    \item[Explanation:]
        The longest alternating subsequence in $nums$ is $[1,17,5,15,5,16,8]$ which has length $7$.

\end{description}

    

\subsection{Brute Force Approach to Longest Alternating Subsequence}

The naiive way to approach this problem is to generate every possible subsequence of nums. 
For each subsequence, we can check if it is alternating iteratively, and if it is, calculate its length.
We keep track of the maximum length of all of the alternating subsequences.

An implementation of the brute force algorithm is given in Figure \ref{fig:las-bf}.

\begin{figure}[H]
    \centering
    \begin{lstlisting}
    def is_alternating(sequence):
        if len(sequence) < 3:
            return True
    
        for i in range(1, len(sequence) - 1):
            if not ((sequence[i - 1] > sequence[i] < sequence[i + 1]) or
                    (sequence[i - 1] < sequence[i] > sequence[i + 1])):
                return False
        return True
    
    def las_bf(nums):
        if not nums:
            return 0
    
        n = len(nums)
        max_length = 1
    
        for i in range(1 << n):
            subsequence = [nums[j] for j in range(n) if (i & (1 << j)) > 0]
            if is_alternating(subsequence):
                max_length = max(max_length, len(subsequence))
    
        return max_length
    \end{lstlisting}
    \caption{Longest Alternating Subsequence Brute Force Python Implementation}
    \label{fig:las-bf}
\end{figure}


In line 18 in Figure \ref{fig:las-bf}, The expression $1 << n$ represents a bitwise left shift operation.
In Python, $<<$ is the left shift operator, and it shifts the binary representation of the number to the left by $n$ positions.
In the context of generating all possible subsequences, $1 << n$ is used to create a bitmask with the rightmost $n$ bits set to $1$.
Each bit in the bitmask corresponds to whether the corresponding element in the array is included or excluded in the current subsequence.
The line "for $i$ in range$(1 << n)$" generates all possible subsequences by iterating through all bitmasks from 0 to $2^n-1$.

\subsection{Complexity Analysis of the Brute Force Approach to Longest Alternating Subsequence}
Let n be the length of nums.
\begin{description}
    \item[Time Complexity:]
        The complexity of generating every possible subsequence is $O(2^n)$.
        Checking if a sequence is alternating is $O(|sequence|)$,
        and getting the length of a subsequence, as well as comparing it to the $max\_length$ are $O(1)$.
        This is overall of order $O(2^n)$.
        
    \item[Space Complexity:] 
        The space complexity is $O(n)$, as we must store a single subarray at a time,
        which in the worst case is of length $n$.
        
    \item[Overall:] Total:\\
        Time Complexity: $O(2^n)$\\
        Space Complexity: $O(n)$
    
\end{description}

Notice that for an aribitrary longest alternating subsequence of length $k$ ending at index $i$,
if there exists exactly one number which can extend the sequence,
the length of the total longest alternating subsequence is guaranteed to be $k+1$.
This shows the optimal substructure property.
Also, when looking for subsequences at index $i$,
we must re-compute all of the subsequences at index $i-1$.
This shows the overlapping subproblems property.

As using memoization on subsequence problems is not space efficient,
we can use tabulation and only store the necessary information rather than an intermediate result for every subsequence.
The following is known as the Auxiliary Arrays solution.

\subsection{Auxiliary Arrays Solution for Longest Alternating Subsequence}
Two auxiliary arrays $inc$ and $dec$, of length $|nums|$ are initialized.
$inc[i]$ contains the length of the longest alternating subsequence of $nums[0:i]$,
where the last element of the subsequence is greater than the previous element.
$dec[i]$ contains the length of the longest alternating subarray of $nums[0:i]$,
where the last element of the subsequence is less than the previous element.

The algorithm iterates through $nums$, considering each element $nums[i]$ and updating the inc and dec arrays based on the following conditions:

\begin{enumerate}
    \item If $nums[i]$ is greater than $nums[j]$ for some previous index $j$, it means the sequence can be extended in an increasing manner.
    In this case, $inc[i]$ is updated to be the maximum of its current value and the length of the longest decreasing subsequence ending at index $j+1$, found in $dec[j+1]$.

    \item If $nums[i]$ is smaller than $nums[j]$ for some previous index $j$, it means the sequence can be extended in a decreasing manner.
    In this case, $dec[i]$ is updated to be the maximum of its current value and the length of the longest increasing subsequence ending at index $j+1$, found in $inc[j+1]$.

    \item The length of the longest alternating subsequence is the maximum value in the $inc$ and $dec$ arrays.
\end{enumerate}


A sample python implementation is shown in Figure \ref{fig:las-auxiliary}.

\begin{figure}[H]
    \centering
    \begin{lstlisting}
    def las_auxiliary(nums):
        n = len(nums)
        
        inc = [1] * n
        dec = [1] * n
    
        for i in range(1, n):
            for j in range(i):
                if nums[i] > nums[j]:
                    inc[i] = max(inc[i], dec[j] + 1)
                elif nums[i] < nums[j]:
                    dec[i] = max(dec[i], inc[j] + 1)

        return max(max(inc), max(dec))
    \end{lstlisting}
    \caption{Longest Alternating Subsequence Auxiliary Arrays Python Implementation}
    \label{fig:las-auxiliary}
\end{figure}

\subsection{Complexity Analysis of the Auxiliary Arrays Approach to Longest Alternating Subsequence}
Let n be the length of nums.
\begin{description}
    \item[Time Complexity:]
        We use a double nested for loop to iterate over $nums$.
        This gives a complexity of $O(n^2)$.
        All other operations are either constant time lookups, updates or $max(a,b)$,
        all of which are $O(1)$.
        
    \item[Space Complexity:] 
        The space complexity is $O(2n)$, which is of order $O(n)$.
        This is because we must store the $inc$ array and the $dec$ array,
        each of which have the same length as $nums$.
        All other variables are stored with constant space complexity.

        
    \item[Overall:] Total:\\
        Time Complexity: $O(n^2)$\\
        Space Complexity: $O(n)$
    
\end{description}


\subsection{Optimized Soluiton to Longest Alternating Subsequence}
Observe that at each step, the max of $inc$ and $dec$ can be found at the current index of the arrays,
so we do not need to store the entire $inc$ and $dec$ arrays.
Instead, we can use an $inc$ and $dec$ variable which will hold the value stored at $inc[i]$ and $dec[i]$ respectively.
This is because a maximum alternating subsqeuence of $nums[0:a]$ will never be of lower length than a maximum alternating subsequence of $nums[0:b]$ if $a > b$.

We can do a single iteration over nums where:

\begin{enumerate}
    \item $inc$ should be set to $dec + 1$, if and only if the last element in the alternating sequence was less than its previous element.
    
    \item $dec$ should be set to $inc + 1$, if and only if the last element in the alternating sequence was greater than its previous element.

\end{enumerate}

A sample python implementation is shown in Figure \ref{fig:las-optimized}.

\begin{figure}[H]
    \centering
    \begin{lstlisting}
    def las_optimized(nums):
        n = len(nums)
    
        inc = 1
        dec = 1
    
        for i in range(1, n):
            if nums[i] > nums[i - 1]:
                inc = dec + 1
            elif nums[i] < nums[i - 1]:
                dec = inc + 1
    
        result = max(inc, dec)
        return result
    \end{lstlisting}
    \caption{Longest Alternating Subsequence Optimzied Python Implementation}
    \label{fig:las-optimized}
\end{figure}

\subsection{Complexity Analysis of the Optimized Approach to Longest Alternating Subsequence}
Let n be the length of nums.

\begin{description}
    \item[Time Complexity:]
        We use a single for loop to iterate over $nums$.
        This gives a complexity of $O(n)$.
        All other operations are either constant time lookups,
        updates or $max(a,b)$, all of which are $O(1)$.
        
    \item[Space Complexity:] 
        The space complexity is $O(1)$ as we only need to store a single value for $inc$ and $dec$.

    \item[Overall:] Total:\\
        Time Complexity: $O(n)$\\
        Space Complexity: $O(1)$
    
\end{description}


\section{Binomial Coefficients}
\begin{description}
    \item[Problem Statement:]
        Given as input positive integers $n$ and $k$ where $n \geq k$, calculate how many different ways you can select $k$ unique examples from a set of size $n$.
        In other words, calculate $C(n,k)$ where: $$C(n,k) = n! / k! ((n-k)!)$$

    \item[Input:] 
        Two positive integers $C(n,k)$
        
    \item[Output:]
        A single positive integer $C(n,k)$
        
    \item[Example:] For:\\
        $n = 4$\\
        $k = 2$\\
        $C(n,k) = 6$

    \item[Explanation:]
        There are 6 unique ways you can choose 2 different examples from a set of size 4.
        
\end{description}

We know that by definition:

$$C(n,k) = C(n-1,k-1) + C(n-1,k)$$

$$C(n,0) = 1$$

$$C(n,n) = 1$$

We can use this as a recursive case and base cases in a brute force solution.

\subsection{Brute Force Approach to Binomial Coefficients}
We can use recursion to arrive at our answer, using the definition as our recursive case and our base cases.

A sample python implementation is shown in Figure \ref{fig:binomial-bf}.

\begin{figure}[H]
    \centering
    \begin{lstlisting}
    def C_bf(n,k):
        if k == 0 or k == n: return 1
        return C_bf(n-1,k-1) + C_bf(n-1,k)
    \end{lstlisting}
    \caption{Binomial Coefficients Brute Force Python Implementation}
    \label{fig:binomial-bf}
\end{figure}

\subsection{Complexity Analysis of the Brute Force Approach to Binomial Coefficients}
\begin{description}
    \item[Time Complexity:]
        For our analysis, in the worst case $k$ is equal to $n/2$.
        This means that at each recursion,
        we create two subproblems.
        We do this $n$ times, giving a total time complexity of $O(2^n)$.

    \item[Space Complexity:] 
        The space complexity is determined by the depth of recursion, which is $O(n)$.

    \item[Overall:] Total:\\
        Time Complexity: $O(2^n)$\\
        Space Complexity: $O(n)$
    
\end{description}

\subsection{Memoization Approach to Binomial Coefficients}
If we look at the trace of calculating $C(4,2)$ in our recursive algorithm, we can see that it is guaranteed to be the sum of $C(3,1)$ and $C(3,2)$, as by definition $C(n,k) = C(n-1,k-1) + C(n-1, k)$.
This shows the problem has the optimal substructure property.
To calculate $C(3,1)$ we must know $C(2,1)$.
This is also true when calculating $C(3,2)$.
This means there is repeated calculations of $C(2,1)$,
showing the overlapping subproblems property.
In larger problems, the number of repeated calculations is vast.
We can use memoization to store each calculation we do in a table called $memo$ to avoid repeating subproblems.
At each step, before continuing with recursion, we check $memo$ to see if the calculation has been done already, and if so, we take the value from the $memo$ table instead of making a recursive call.

A sample python implementation is shown in Figure \ref{fig:binomial-memo}.

\begin{figure}[H]
    \centering
    \begin{lstlisting}
    def C_memo(n,k,memo={}):
        if k == 0 or k == n: return 1
    
        if (n,k) in memo:
            return memo[(n,k)]
        
        result = C_memo(n-1,k-1,memo) + C_memo(n-1,k,memo)
        memo[(n,k)] = result
        return result
    \end{lstlisting}
    \caption{Binomial Coefficeints Memoization Python Implementation}
    \label{fig:binomial-memo}
\end{figure}

\subsection{Complexity Analysis of the Memoization Approach to Binomial Coefficients}
\begin{description}
    \item[Time Complexity:]
        The memoization table ensures that each unique subproblem is solved only once.
        In the case of "$n$ choose $k$," there are $O(n * k)$ unique subproblems because the parameters $n$ and $k$ can take values from $0$ to $n$.
        The rest of the table lookups are expected constant time as we are using a dictionary for the $memo$ table.

    \item[Space Complexity:] 
        We must store the $memo$ table, which is a dictionary of size $O(n * k)$ as $n$ and $k$ can take values from $0$ to $n$.

    \item[Overall:] Total:\\
        Time Complexity: $O(n * k)$\\
        Space Complexity: $O(n * k)$
    
\end{description}

\subsection{Tabulation Approach to Binomial Coefficients}
Instead of recursively creating the table by searching the entire tree of possible $n$ and $k$ values, we can use tabulation to fill in a table from which we can extract our answer.
Consider a table $dp$ with dimensions ($n+1$) x ($k+1$), where $dp[i][j]$ in the table contains the result of $C(i,j)$
We could build this table up starting from our base cases:
$$C(n,0) = 1$$

$$C(n,n) = 1$$

, until we have a solution to $C(n,k)$.

So for the calculation of $C(4,2)$ for example, we would initialize the following table:

\begin{table}[htbp]
    \centering
    \begin{tabular}{|c|c|c|c|}
        \hline
          & \textbf{0} & \textbf{1} & \textbf{2} \\
        \hline
        \textbf{0} & 1 & 0 & 0 \\
        \hline
        \textbf{1} & 1 & 1 & 0 \\
        \hline
        \textbf{2} & 1 &   & 1 \\
        \hline
        \textbf{3} & 1 &   &  \\
        \hline
        \textbf{4} & 1 &   &   \\
        \hline
    \end{tabular}
\end{table}
Now, we can use $C(n,k) = C(n-1,k-1) + C(n-1,k)$ to fill the table.

For example, to get $dp[2][1]$ $(C(2,1))$, we take $dp[1][0] + dp[1][1] = 2$.

Do this for the entire table as follows:

\begin{table}[htbp]
    \centering
    \begin{tabular}{|c|c|c|c|}
        \hline
          & \textbf{0} & \textbf{1} & \textbf{2} \\
        \hline
        \textbf{0} & 1 & 0 & 0 \\
        \hline
        \textbf{1} & 1 & 1 & 0 \\
        \hline
        \textbf{2} & 1 & 2 & 1 \\
        \hline
        \textbf{3} & 1 & 3 & 3 \\
        \hline
        \textbf{4} & 1 & 4 & 6 \\
        \hline
    \end{tabular}
\end{table}

Until we arrive at $dp[n][k]$ which will give the answer of $C(n,k)$.

A sample python implementation is shown in Figure \ref{fig:binomial-dp}.

\begin{figure}[H]
    \centering
    \begin{lstlisting}
    def C_dp(n,k):
        dp = [[0] * (k+1) for _ in range(n+1)]
    
        #Fill in the base cases
        for i in range(n+1):
            dp[i][0] = 1
            dp[i][min(i,k)]=1
    
        #Fill in the rest of the table using the definition of C(n,k)
        for i in range(1,n+1):
            for j in range(1,min(i,k)+1):
                dp[i][j] = dp[i-1][j-1] + dp[i-1][j]
                
        return dp[n][k]
    \end{lstlisting}
    \caption{Binomial Coefficients Tabulation Python Implementation}
    \label{fig:binomial-dp}
\end{figure}

We do not fill the table where $k > n$.
This because it is impossible to choose $k$ unique examples from a set of size $n$.
This is done by only iterating to $min(i,k)$ in the inner loop.

\subsection{Complexity Analysis of the Tabulation Approach to Binomial Coefficients}

\begin{description}
    \item[Time Complexity:]
        We have to fill in a table of size $O(n * k)$ with values, and each value is obtained through a constant time lookup and addition.

    \item[Space Complexity:] 
        We must store the $dp$ table, which is a 2D array of size $O(n * k)$ as $n$ and $k$ can take values from $0$ to $n$.
        
    \item[Overall:] Total:\\
        Time Complexity: $O(n * k)$\\
        Space Complexity: $O(n * k)$
    
\end{description}

\subsection{Optimzied Tabulation Approach to Binomial Coefficeints}
Notice that we can calculate the values in any row $r$ using the values in row $r-1$.
Hence, we do not need to store the entire $dp$ table in memory, only two rows at a time.
The current row $r$, and the previous row $r-1$.
This reduces space complexity from $O(n * k)$ to $O(k)$.
Notice also that $C(n,k) = C(n,n-k)$.
We can therefore set $k = n-k$ if $k > n-k$ and $n-k$ is positive before we start our algorithm, and our result will be the same.
This reduces the time complexity from $O(n * k)$ to $O(n * min(k,n-k))$.

A sample python implementation is shown in Figure \ref{fig:binomial-optimized}.

\begin{figure}[H]
    \centering
    \begin{lstlisting}
    def C_optimized(n,k):
        if k > n-k and n-k >= 0:
            k = n-k
        oldRow = [0] * (k+1)
        oldRow[0] = 1
    
        for i in range(1,n+1):
            newRow = [0] * (k+1)
            newRow[0] = 1
            for j in range(1,min(i,k)+1):
                newRow[j] = oldRow[j-1] + oldRow[j]
    
            oldRow = newRow
              
        return newRow[k]
    \end{lstlisting}
    \caption{Binomial Coefficients Optimized Python Implementation}
    \label{fig:binomial-optimized}
\end{figure}



\section{Longest Common Subsequence} \label{section:lcs}
\begin{description}
    \item[Problem Statement:]
        Given two strings, return the length of their longest common subsequence\footnote{Subsequences do not have to be contiguous.}.

    \item[Input:] 
        Two strings $text1$ and $text2$.
    \item[Output:] 
        An integer $lcs$, the longest common subsequence of $text1$ and $text2$.
        
    \item[Example:] For:\\
        $text1 = "abcde"$\\
        $text2 = "ace"$\\
        $lcs = 3$

    \item[Explanation:]
        The longest common subsequence of $"abcde"$ and $"ace"$ is $"ace"$, which is of length $3$.
   
\end{description}

\subsection{Longest Common Subsequence Brute Force}
The naiive way to find the longest common subsequence of two strings is to generate all possible subsequences of both strings.
We initialize a $lcs$ variable to store the length of the longest common subsequence.
We then iterate over each list of subsequences and report any common subsequences we find, updating $lcs$ accordingly.
Finally we return $lcs$.

\subsection{Complexity Analysis of the Brute Force Approach to Longest Common Subsequence}
For the calculation of worst case time and space complexity, we assume that both strings are equal in length, and that length is denoted by $n$.

\begin{description}
    \item[Time Complexity:]
        The time complexity of generating every subsequence of a string of length $n$ is $O(2^n)$,
        and doing this for both strings is $O(2 * (2^n))$ which is $O(2^n+1)$,
        as for each character we decide if we include it in or exclude it from the substring.
        The time complexity of iterating over two lists of substrings both of size $O(2^n)$ and looking for matches is $O((2^n)^2)$ which is equal to $O(2^{2n})$.
        This is still of order $O(2^n)$.

        
    \item[Space Complexity:]
        The space complexity of storing every subsequence of a string of length $n$ is $O(2^n)$.
        As we must do this for two strings, the total space complexity required is $O(2 * 2^n)$ which is of order $O(2^n)$.

    \item[Overall:] Total:\\
        Time Complexity: $O(2^n)$\\
        Space Complexity: $O(2^n)$
    
\end{description}

Notice that:

OBSERVATION 1: If we have a common character at the current position, such as position 1 in:

$$text1=abcde, text2=ace$$

The solution will be $1 + lcs(bcde,ce)$\footnote{Remove the common character from both texts, add 1 and recurse. The reason we add 1 is because each common character contributes 1 length to our final output.}.
This shows the problem has the optimal substructure property.

OBSERVATION 2: If there is no common character at the current position such as in:

$$text1=bcde, text2=ce$$

The solution will be $max(lcs(cde,ce), lcs(bcde,e))$\footnote{Remove the first character from either the first or second text, recurse on both cases.}.
As we have the chance to recurse on the same two substrings multiple times, the problem has the overlapping subproblems property.
\subsection{Tabulation Approach to Longest Common Subsequence}
We can use these two cases to solve this problem by tabulation.
We create a table called $dp$ with $i+1$ rows and $j+1$ columns, where $i$ and $j$ are the lengths of $text1$ and $text2$ respectively.
Each row and column of $dp$ represents a character in $text1$ and $text2$.

\begin{table}[htbp]
    \centering
    \begin{tabular}{|c|c|c|c|}
        \hline
          & \textbf{a} & \textbf{c} & \textbf{e} \\
        \hline
        \textbf{a} &   &   &   \\
        \hline
        \textbf{b} &   &   &    \\
        \hline
        \textbf{c} &   &   &    \\
        \hline
        \textbf{d} &   &   &   \\
        \hline
        \textbf{e} &   &   &    \\
        \hline
    \end{tabular}
\end{table}

In this table, for example,
$dp[0][0]$ will represent $lcs(abcde,ace) \rightarrow$  the solution.
$dp[3][1]$ will represent $lcs(de,ce)$ ect..
i.e. $dp[i][j]$ represents the longest common subseqence of $text1[i:]$ and $text2[j:]$.

We fill the table starting from the bottom right as follows:

\begin{enumerate}
    \item If the row and column have the same label, we have found a common character.

    As per OBSERVATION 1, we fill the space with $1 + dp[i+1][j+1]$.
    
    \item If the row and col do not have the same label, we must use OBSERVATION 2:

    Fill the space with $max(dp[i][j+1], dp[i+1][j])$.
    
    \item If we go out of bounds, we simply take zero\footnote{For ease of code, the grid is ($i+1$) x ($j+1$) and initialized to all zeros, so we don't have to check for going out of bounds.}.
\end{enumerate}

Our solution will be located at $dp[0][0]$, which represents $lcs(text1,text2)$.

\begin{table}[htbp]
    \centering
    \begin{tabular}{|c|c|c|c|}
        \hline
          & \textbf{a} & \textbf{c} & \textbf{e} \\
        \hline
        \textbf{a} & 3 & 2 & 1 \\
        \hline
        \textbf{b} & 2 & 2 & 1  \\
        \hline
        \textbf{c} & 2 & 2 & 1  \\
        \hline
        \textbf{d} & 1 & 1 & 1 \\
        \hline
        \textbf{e} & 1 & 1 & 1  \\
        \hline
    \end{tabular}
\end{table}



A sample python implementation is shown in Figure \ref{fig:lcs-dp}.

\begin{figure}[H]
    \centering
    \begin{lstlisting}
    def lcs(text1, text2):
        dp=[[0 for j in range(len(text2)+1)] for i in range(len(text1)+1)]
    
        for i in range(len(text1)-1,-1,-1):
            for j in range(len(text2)-1,-1,-1):
                if text1[i] == text2[j]:
                    dp[i][j] = 1 + dp[i+1][j+1]
                else:
                    dp[i][j] = max(dp[i][j+1], dp[i+1][j])
    
        return dp[0][0]
    \end{lstlisting}
    \caption{Longest Common Subsequence Tabulation Python Implementation}
    \label{fig:lcs-dp}
\end{figure}

\subsection{Complexity Analysis of the Tabulation Approach to Longest Common Subsequence}

Let $i$ be the length of $text1$, and $j$ be the length of $text2$.

\begin{description}
    \item[Time Complexity:]
        The time complexity of filling an $i * j$ table with values, where each value is derived from a constant time lookup and a constant time addition or max function is $O(i * j)$.

    \item[Space Complexity:] 
        The space complexity of storing an $i * j$ table is $O(i * j)$.

        
    \item[Overall:] Total:\\
        Time Complexity: $O(i * j)$\\
        Space Complexity: $O(i * j)$
    
\end{description}

NOTE THAT SINCE EACH ROW OF THIS TABLE CAN BE COMPUTED USING THE PREVIOUS ROW ONLY, THE SAME TWO-ROW METHOD CAN BE USED AS IN BINOMIAL COEFFICIENTS

\section{Longest Palindromic Subsequence}
\begin{description}
    \item[Problem Statement:]
        Given a string, return the length of the longest subsequence\footnote{Subsequences do not have to be continuous.} of the string which is a palindrome\footnote{A palindrome is a string which is identical to itself when reversed (Example: "racecar").}.
    
    \item[Input:]
        A string $s$.
        
    \item[Output:]
        An integer $lps$ which represents the longest palondromic subsequence of $s$.
        
    \item[Example:] For:\\
        $s = "babbb"$\\
        $lps = 4$
        
    \item[Explanation:]
        $"bbbb"$ is the longest palindromic subsequence of $s$. It has length $4$.
\end{description}

\subsection{Tabulation Approach to Longest Palindromic Subsequence}
This problem is simply a special case of Longest Common Subsequence.
The longest palindromic subsequence of $s$ is simply the longest common subsequence of $s$ and $reverse(s)$.
Therefore the same logic applies. See Section \ref{section:lcs}.

A sample python implementation is shown in Figure \ref{fig:lps-dp}.
For the implementation of the lcs subroutine, see Figure \ref{fig:lcs-dp}.

\begin{figure}[H]
    \centering
    \begin{lstlisting}
    def lps(s):
        return lcs(s,s[::-1])
    \end{lstlisting}
    \caption{Longest Palindromic Subsequence Python Implementation}
    \label{fig:lps-dp}
\end{figure}

\section{Longest Contiguous Palindromic Substring}
\begin{description}
    \item[Problem Statement:]
        Given a string, return the longest palindromic substring\footnote{Substrings must be contiguous.} of the string.

    \item[Input:] 
        A string $s$.
        
    \item[Output:] 
        A string $lpcs(s)$, which is the longest palindromic substring of $s$.
        
    \item[Example:] For:\\
        $s = "aaaabbaa"$\\
        $lpcs(s) = "aabbaa"$

    \item[Explanation:]
        The longest palindromic substring of $"aaaabbaa"$ is $"aabbaa"$.
        Note that here we are asked for the substring itself, rather than the length of the substring.
   
\end{description}

\subsection{Brute Force Approach to Longest Contiguous Palindromic Substring}
If we were to tackle this problem the brute force way, we would have to iterate over all substrings of s, filtering out non palindromes along the way,
and keeping track of the longest palindromic substring found so far. 
We would then return the longest palindromic substring of s.

\subsection{Complexity Analysis of the Brute Force Approach to Longest Contiguous Palindromic Substring}
Let n be the length of s.

\begin{description}
    \item[Time Complexity:]
        Iterating over all substrings of a string of length $n$ has a time complexity of $O(n^2)$.
        Checking if a substring is a palindrome has a time complexity of $O(n)$.
        This gives the brute force approach an overall time complexity of $O(n^3)$.
        
    \item[Space Complexity:] 
        The space complexity of generating all substrings and storing them in a list is also $O(n^3)$
        because we have $O(n^2)$ substrings, each of which has a maximum length of $O(n)$.
        However, in our algorithm it is possible to iterate over the subsrtings of $s$ in place using a double for loop
        \footnote{See Figure \ref{fig:max-subarray-sum-bf} for an example of how to do this.}, we can store one substring at a time,
        which has a maximum length of $n$. This brings the space complexity down to $O(n)$.
        
    \item[Overall:] Total:\\
        Time Complexity: $O(n^3)$\\
        Space Complexity: $O(n)$
    
\end{description}
Notice that if a string $p$ is a palindrome, and $x$ is a character, the string $xpx$ is guaranteed to be a palindrome.
This is proof of optimal substructure.
The problem of determining whether a substring is a palindrome or not can be reduced to smaller subproblems.
For example, when checking if $s[i:j]$ is a palindrome, we often need to check if $s[i+1:j-1]$ is a palindrome,
which overlaps with other similar subproblems.
Therefore, we can use dynamic programming principles to optimize the solution to this problem.
\subsection{Tabulation Approach to Longest Contiguous Palindromic Substring.}
We can create a table $dp$ where $dp[i][j]$ is 1 if $s[i:j+1]$ is a palindrome, else 0. (eg: if $dp[1][3] = 1$, $s[1:4] (\text{"aaa" in our example string})$ is a palindrome.)
We start with a table $dp$ which is a 2D matrix of size $|s| * |s|$.
We can initialize all fields where $i == j$ to 1, as single letters are palindromes.

\begin{table}[htbp]
    \centering
    \begin{tabular}{|c|c|c|c|c|c|c|c|c|}
        \hline
          & \textbf{a} & \textbf{a} & \textbf{a} & \textbf{a} & \textbf{b} & \textbf{b} & \textbf{a} & \textbf{a} \\
        \hline
        \textbf{a} & 1 &  &  &  &  &  &  &  \\
        \hline
        \textbf{a} &  & 1 &  &  &  &  &  &  \\
        \hline
        \textbf{a} &  &  & 1 &  &  &  &  &  \\
        \hline
        \textbf{a} &  &  &  & 1 &  &  &  &  \\
        \hline
        \textbf{b} &  &  &  &  & 1 &  &  &  \\
        \hline
        \textbf{b} &  &  &  &  &  & 1 &  &  \\
        \hline
        \textbf{a} &  &  &  &  &  &  & 1 &  \\
        \hline
        \textbf{a} &  &  &  &  &  &  &  & 1 \\
        \hline
    \end{tabular}
\end{table}

We can initialize all fields where $j = i+1$ and $s[i] = s[j]$ to 1 as all pairs of the same letter are palindromes.
This handles the case where the palindrome has an even amount of characters (meaning the center of the palindrome is a character pair).


\begin{table}[htbp]
    \centering
    \begin{tabular}{|c|c|c|c|c|c|c|c|c|}
        \hline
          & \textbf{a} & \textbf{a} & \textbf{a} & \textbf{a} & \textbf{b} & \textbf{b} & \textbf{a} & \textbf{a} \\
        \hline
        \textbf{a} & 1 & 1 &  &  &  &  &  &  \\
        \hline
        \textbf{a} &  & 1 & 1 &  &  &  &  &  \\
        \hline
        \textbf{a} &  &  & 1 & 1 &  &  &  &  \\
        \hline
        \textbf{a} &  &  &  & 1 & 0 &  &  &  \\
        \hline
        \textbf{b} &  &  &  &  & 1 & 1 &  &  \\
        \hline
        \textbf{b} &  &  &  &  &  & 1 & 0 &  \\
        \hline
        \textbf{a} &  &  &  &  &  &  & 1 & 1 \\
        \hline
        \textbf{a} &  &  &  &  &  &  &  & 1 \\
        \hline
    \end{tabular}
\end{table}

Now, notice that if a string $p$ is a palindrome, and $x$ is a character, the string $xpx$ is guaranteed to be a palindrome.
Using this, for all substrings $s'$ of length 3, we check if the start and end are the same letter, and the middle is a palindrome (we know from the existing entries in the table).
If so, we know that $s'$ is a palindrome, so we can set $dp[i][j]$ to 1.
We can put a 1 in $dp[i][j]$ the table if and only if:

\begin{enumerate}
    \item If $s[i] == s[j]$ (the starting character is equal to the ending character).

    AND

    \item If $dp[i+1][j-1]$ == 1 (the string $p$ in between $i$ and $j$ is already known to be a palindrome).

\end{enumerate}

We will end up with a table as follows:

\begin{table}[htbp]
    \centering
    \begin{tabular}{|c|c|c|c|c|c|c|c|c|}
        \hline
          & \textbf{a} & \textbf{a} & \textbf{a} & \textbf{a} & \textbf{b} & \textbf{b} & \textbf{a} & \textbf{a} \\
        \hline
        \textbf{a} & 1 & 1 & 1 &  &  &  &  &  \\
        \hline
        \textbf{a} &  & 1 & 1 & 1 &  &  &  &  \\
        \hline
        \textbf{a} &  &  & 1 & 1 & 0 &  &  &  \\
        \hline
        \textbf{a} &  &  &  & 1 & 0 & 0 &  &  \\
        \hline
        \textbf{b} &  &  &  &  & 1 & 1 & 0 &  \\
        \hline
        \textbf{b} &  &  &  &  &  & 1 & 0 & 0 \\
        \hline
        \textbf{a} &  &  &  &  &  &  & 1 & 1 \\
        \hline
        \textbf{a} &  &  &  &  &  &  &  & 1 \\
        \hline
    \end{tabular}
\end{table}

Do this for all lengths from $3 \rightarrow len(s) - 1$.
We will end up with the following table.

\begin{table}[htbp]
    \centering
    \begin{tabular}{|c|c|c|c|c|c|c|c|c|}
        \hline
          & \textbf{a} & \textbf{a} & \textbf{a} & \textbf{a} & \textbf{b} & \textbf{b} & \textbf{a} & \textbf{a} \\
        \hline
        \textbf{a} & 1 & 1 & 1 & 1 & 0 & 0 & 0 & 0 \\
        \hline
        \textbf{a} &  & 1 & 1 & 1 & 0 & 0 & 0 & 0 \\
        \hline
        \textbf{a} &  &  & 1 & 1 & 0 & 0 & 0 & 1 \\
        \hline
        \textbf{a} &  &  &  & 1 & 0 & 0 & 1 & 0 \\
        \hline
        \textbf{b} &  &  &  &  & 1 & 1 & 0 & 0 \\
        \hline
        \textbf{b} &  &  &  &  &  & 1 & 0 & 0 \\
        \hline
        \textbf{a} &  &  &  &  &  &  & 1 & 1 \\
        \hline
        \textbf{a} &  &  &  &  &  &  &  & 1 \\
        \hline
    \end{tabular}
\end{table}


From the table we can deduce all palindromic substrings, including the longest palindromic substring.
To get the longest palindromic substring from the table, we track $start$ and $max\_length$, which get updated every time a new palindrome is found.
This works because palindromes are found from shortest to longest, so every new palindrome found is the maximum length palindrome.
We can then simply return a slice of the input string as follows: $s[start,start+max_length]$.
This will yield the maximum length palindromic substring.

A sample python implementation is shown in Figure \ref{fig:lpcs-dp}.

\begin{figure}[H]
    \centering
    \begin{lstlisting}
    def lpcs(s):
        n = len(s)
        # Single character strings are palindromes, so we can return s
        if n <=1: return s
    
        # Create a n*n table of zeros
        dp = [[0] * n for _ in range(n)]
    
        # Initialize all single character substrings to 1

        # Single character substrings start and end at the same index, so we locate them with dp[i][i]
        for i in range(n):
            dp[i][i] = 1
    
        # Since we are looking for the longest palindromic substring itself and not the length,
        # we can track the starting position and max_length for convenience
        start = 0
        max_length = 1
    
        # Initialize all pairs of identical characters to 1
        for i in range(n-1):
            if s[i] == s[i+1]:
                dp[i][i+1] = 1
                start, max_length = i,2
    
        # For all substrings with length >=3, starting at length 3, 
        # set dp[i][j] to 1 if start and end are the same letter
        # and the middle is a palindrome
                
        for length in range(3, n+1):
            for i in range(n - length + 1):
                j = i + length - 1
                if dp[i+1][j-1] and s[i] == s[j]:
                    dp[i][j] = 1
                    start, max_length = i, length
    
        return s[start:start+max_length]
    \end{lstlisting}
    \caption{Longest Contiguous Palindromic Substring Tabulation Python Implementation}
    \label{fig:lpcs-dp}
\end{figure}

\subsection{Complexity Analysis of the Tabulation Approach to Longest Contiguous Palindromic Subsequence}
Let $n$ be the length of $s$.

\begin{description}
    \item[Time Complexity:]
        The time complexity to build an $n * n$ table,
        where the values are deduced from a constant time check and a constant time lookup is $O(n^2)$. 

    \item[Space Complexity:] 
        The space complexity to store an $n * n$ table is $O(n^2)$.

        
    \item[Overall:] Total:\\
        Time Complexity: $O(n^2)$\\
        Space Complexity: $O(n^2)$
    
\end{description}

\section{The Needleman-Wunsch Algorithm}
This famous dynamic programming algorithm is used for global alignment of DNA and protein sequences.

\begin{description}
    
    \item[Problem Statement:]
        Given two character sequences $seq1$ and $seq2$,
        place zero or more 'gap's in $seq1$ and/or $seq2$ such that the maximum number of characters in $seq1$ and $seq2$ are 'aligned'.
        Characters $X$ and $Y$ are 'aligned' when $X == Y$ and the index of $X$ in $seq1$ is exactly equal to the index of $Y$ in $seq2$.
        
    \item[Input:] 
        Two strings $seq1$ and $seq2$.
        
    \item[Output:]
        Two strings $align1$ and $align2$.
        
    \item[Example:] For:\\
        $seq1 = "ATGCT"$\\
        $seq2 = "AGCT"$\\
        $align1 = "ATGCT"$\\
        $align2 = "A-GCT"$

    \item[Explanation:]
        Globally Aligned sequences:\\
        A T G C T\\
        A - \ G C T
\end{description}

The trick is to find an efficient way to place gaps in the sequences to maximise the amount of globally aligned letters.
We can do this with tabulation:

Create a matrix of size: $(len(seq1) + 1)$ x $(len(seq2) + 1)$ as follows:

\begin{table}[htbp]
    \centering
    \begin{tabular}{|c|c|c|c|c|c|c|}
        \hline
          &  & \textbf{A} & \textbf{T} & \textbf{G} & \textbf{C} & \textbf{T} \\
        \hline
         &  &  &  &  &  &  \\
        \hline
        \textbf{A} &  &  &  &  &  &  \\
        \hline
        \textbf{G} &  &  &  &  &  &  \\
        \hline
        \textbf{C} &  &  &  &  &  &  \\
        \hline
        \textbf{T} &  &  &  &  &  &  \\
        \hline
    \end{tabular}
\end{table}


Now use the following scheme to fill in the table:

$Match: 1$

$Mismatch: -1$

$GAP: -2$

\subsection{Initialization of the Needleman-Wunsch Table}

Starting with 0 at $(0,0)$, fill the extra row and column with progressive GAP penalties as follows:

\begin{table}[htbp]
    \centering
    \begin{tabular}{|c|c|c|c|c|c|c|}
        \hline
          &  & \textbf{A} & \textbf{T} & \textbf{G} & \textbf{C} & \textbf{T} \\
        \hline
        & 0 & -2 & -4 & -6 & -8 & -10 \\
        \hline
        \textbf{A} & -2 &  &  &  &  &  \\
        \hline
        \textbf{G} & -4 &  &  &  &  &  \\
        \hline
        \textbf{C} & -6 &  &  &  &  &  \\
        \hline
        \textbf{T} & -8 &  &  &  &  &  \\
        \hline
    \end{tabular}
\end{table}

\subsection{Filling the Needleman-Wunsch Table}

Now starting at $(1,1)$, and going row-wise, fill each cell with the max of the following:

\begin{enumerate}
    \item Value from left + $GAP$
    
    \item Value from above + $GAP$
    
    \item Value from diagonal + ($Match$ or $Mismatch$) [depending on wether the row and column label are the same letter]

\end{enumerate}

So, for the $(1,1)$ cell, we fill it with $max(-4,-4,1) = 1$.

And for the $(1,2)$ cell, we fill it with $max(-1,-6,-3) = -1$.

And so on...

Until we get:

\begin{table}[htbp]
    \centering
    \begin{tabular}{|c|c|c|c|c|c|c|}
        \hline
          &  & \textbf{A} & \textbf{T} & \textbf{G} & \textbf{C} & \textbf{T} \\
        \hline
        & 0 & -2 & -4 & -6 & -8 & -10 \\
        \hline
        \textbf{A} & -2 & 1 & -1 & -3 & -5 & -7 \\
        \hline
        \textbf{G} & -4 & -1 & 0 & 0 & -2 & -4 \\
        \hline
        \textbf{C} & -6 & -3 & -2 & -1 & 1 & -1 \\
        \hline
        \textbf{T} & -8 & -5 & -2 & -3 & -1 & 2 \\
        \hline
    \end{tabular}
\end{table}

\subsection{Traceback in the Needleman-Wunsch Table}

Starting from the bottom-right (which will always be the highest value in the matrix),
continue the following until $(0,0)$ is reached:

\begin{itemize}
    \item If $row$ and $col$ labels match, go diagonally top-left.

    \item Else, go to $max(left,above,diagonal)$.
    
    \item Each time we go diagonally, we can align the $row$ and $col$ labels.

    \item Each time we go left, we put a gap in $seq2$ at that index.

    \item Each time we go up, we put a gap in $seq1$ at that index.

\end{itemize}

Note that the aligned sequences will be written from right to left.

A sample python implementation is given in Figure \ref{fig:needleman-wunsch}

\begin{figure}[H]
    \centering
    \begin{lstlisting}
    def needleman_wunsch(seq1, seq2, match=1, mismatch=-1, gap=-2):
        # Initialize the scoring matrix
        m, n = len(seq2), len(seq1)
        score = [[0] * (n + 1) for _ in range(m + 1)]
    
        # Initialize the first row and column
        for i in range(m + 1):
            score[i][0] = i * gap
        for j in range(n + 1):
            score[0][j] = j * gap
    
        # Fill in the scoring matrix
        for i in range(1, m + 1):
            for j in range(1, n + 1):
                match_mismatch = match if seq2[i - 1] == seq1[j - 1] else mismatch
                diagonal = score[i - 1][j - 1] + match_mismatch
                horizontal = score[i][j - 1] + gap
                vertical = score[i - 1][j] + gap
                score[i][j] = max(diagonal, horizontal, vertical)
    
        # Traceback to find the alignment
        align2, align1 = "", ""
        i, j = m, n
        while i > 0 or j > 0:
            if i > 0 and j > 0 and score[i][j] == score[i - 1][j - 1] + (match if seq2[i - 1] == seq1[j - 1] else mismatch):
                align2 = seq2[i - 1] + align2
                align1 = seq1[j - 1] + align1
                i -= 1
                j -= 1
            elif i > 0 and score[i][j] == score[i - 1][j] + gap:
                align2 = seq2[i - 1] + align2
                align1 = "-" + align1
                i -= 1
            else:
                align2 = "-" + align2
                align1 = seq1[j - 1] + align1
                j -= 1
    
        return align1, align2
    
    sequence1 = "ATGCT"
    sequence2 = "AGCT"
    align1, align2 = needleman_wunsch(sequence1, sequence2)
    \end{lstlisting}
    \caption{The Needleman Wunsch Algorithm Python Implementation}
    \label{fig:needleman-wunsch}
\end{figure}



\section{The Smith-Waterman Algorithm}

