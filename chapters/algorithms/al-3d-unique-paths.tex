So far we have seen that Unique Paths and Min Path Sum can be solved for 2D array inputs.
These problems can also be trivially solved for 1D array inputs, and single scalar inputs.
But what if the input to the problem is not a 2D array but a 3D array? (But the constraint which prevents negative progress is still in place).

\subsection{Tabulation Approach to 3D-Unique Paths}
Consider an $m$ x $n$ x $k$ array called $input$, where we start at the top left at index $(0,0,0)$ (denoted $R$), and our target is in the bottom right at index $(m-1, n-1, k-1)$ (denoted $F$).
We can only move down at the current depth, right at the current depth, or "in" (meaning we increase the depth we are at).
The problem, like before, is to find the number of distinct paths from $R$ to $F$.
We can use a similar approach as with the 2D problem, but this time dp will need to be a 3D array to store the intermediate values.
Instead of summing the value to the right and below to get our current value, we sum the value to the right, below, and at the next depth level.
The value at index $dp[i][j][t]$ represents the number of unique paths from $input[i][j][t]$ to $input[m-1][n-1][k-1]$.
We will still initialize $dp[m-1][n-1][k-1]$ to 1.
To calculate the value at $dp[i][j][t]$, we will sum up $dp[i+1][j][t]$, $dp[i][j+1][t]$ and $dp[i][j][t+1]$, unless the value is at a border,
in which case we we will exclude that border from the sum, because going further in that direction is not possible\footnote{eg, if $i+1 = m$, we exclude $dp[i+1][j][t]$ from the sum as $dp[i+1][j][t]$ is out of bounds.}.
The value at $dp[0][0][0]$ will be our solution.

A sample python implementation is shown in Figure \ref{fig:3d-unique-paths}.

\begin{figure}[H]
    \centering
    \begin{lstlisting}
    def unique_paths_3d(m, n, k):
        dp = [[[0 for _ in range(k)] for _ in range(n)] for _ in range(m)]

        dp[m - 1][n - 1][k - 1] = 1

        for i in range(m - 1, -1, -1):
            for j in range(n - 1, -1, -1):
                for t in range(k - 1, -1, -1):
                    if i + 1 < m:
                        dp[i][j][t] += dp[i + 1][j][t]
                    if j + 1 < n:
                        dp[i][j][t] += dp[i][j + 1][t]
                    if t + 1 < k:
                        dp[i][j][t] += dp[i][j][t + 1]

        return dp[0][0][0]
    \end{lstlisting}
    \caption{3D Unique Paths Python Implementation}
    \label{fig:3d-unique-paths}
\end{figure}
\subsection{Complexity Analysis of 3D-Unique-Paths}

\begin{description}
    \item[Time Complexity:]
        We have proved that the time complexity of 2D Unique Paths is $O(m * n)$ in Section \ref{ca-unique-paths}.
        Similarly, we can see that the time complexity of 3D Unique Paths is $O(m * n * k)$.
        
        
    \item[Space Complexity:] 
        We have proved that the space complexity of 2D Unique Paths is $O(m * n)$ in Section \ref{ca-unique-paths}.
        Similarly, we can see that the space complexity of 3D Unique Paths is $O(m * n * k)$.
            
        
    \item[Overall:] Total:\\
        Time Complexity: $O(m * n * k)$\\
        Space Complexity: $O(m * n * k)$
\end{description}

\subsection{Note on this Approach}
We can trivially convert this 3D Unique Paths algorithm into a 3D Min Path Sum solution, and in fact we can convert any 2D problem with the constraint that negative progress is not allowed into a 3D problem using this approach.



