\begin{description}
    \item[Problem Statement:]
        Given a list of denominations of coins $D$ and an integer amount $a$, compute the minimum amount of coins (where each coin's denomination $\in D$) needed to sum exactly to the given amount $a$.
        
    \item[Input:] 
        An integer array $D$ of possible coin denominations, and an integer amount $a$.
        
    \item[Output:] 
        An integer $r$, which represents the minimum amount of coins with denominations $\in D$ needed in order to sum exactly to $a$. If this cannot be done, return $-1$.
        
    \item[Example:]
        For: 

        $D = [1, 5, 10, 20]$

        $a = 115$

        $r = 7$

    \item[Explanation:]
        The minimum amount of coins with denominations in $D$ needed to sum to $a$ is 7.

        These coins are: $[20,20,20,20,20,10,5]$


\end{description}

The problem appears trivial at first glance. One may be tempted to use a greedy method as follows:

\subsection{Greedy Approach to the Coin Change Problem}

Algorithm~\ref{algo:coin-change-greedy} shows a greedy approach to the coin change problem.

\begin{algorithm}[H]
    \caption{Greedy Approach to the Coin Change Problem}
    \label{algo:coin-change-greedy}
    \KwIn{List of denominations of coins $D$ and an amount $a$}
    \KwOut{$r$, The minimum number of coins required to make change for $a$}
    Sort $D$ in ascending order\;
    $r \leftarrow 0$\;
    $total \leftarrow 0$\;
    \While{$\text{total} < a$}{
        \If{$|D| = 0$}{
            \KwRet{-1}\;
        }
        \If{$\text{total} + D[-1] > a$}{
            $D.\text{pop()}$\;
        }
        \Else{
            $\text{total} \leftarrow \text{total} + D[-1]$\;
            $r \leftarrow r + 1$\;
        }
    }
    \KwRet{$r$}
\end{algorithm}

In Algorithm~\ref{algo:coin-change-greedy}, we always choose the coin with the largest value which will not make the total exceed a.

\subsection{Optimality of the Greedy Approach to the Coin Change Problem}

This algorithm is not optimal, and we can prove this by counter-example.\\
A counter example is $D = [5,4,3,2,1]$, $a = 7$.\\
Given these inputs, the greedy result is: $r1 = 3$  ([5,1,1]).\\
The optimal solution for these inputs is: $r2 = 2$ ([4,3]).\\
We see that $r1 > r2$, meaning the greedy approach does not find the miminized solution.\\

\subsection{Correctness of the Greedy Approach to the Coin Change Problem}

The greedy algorithm is also not correct, and we can prove this by another counter-example.\\
A counter example is $D = [4,3]$, $a = 6$.\\
Given these inputs, the greedy result is: $r1 = -1$  ([4]).\\
The optimal solution for these inputs is: $r2 = 2$ ([3,3]).\\
We see that the greedy approach fails when a solution is indeed possible, as shown by $r2$.\\
Since we have shown that the greedy approach is neither correct nor optimal, we move on to the brute force solution.\\


\subsection{Brute Force Approach to the Coin Change Problem}

The brute force approach to the coin change problem involves generating all possible coin combinations, and checking if any of them sum exactly to $a$.
Of the ones that do, we return the minimum length.\\
To try all possible coin combinations, we can subtract each coin denomination $c \in D$ from $a$, as long as $a - c >= 0$.\\
We can repeat this step for each result obtained from this calculation (replacing $a$ with the intermediate result), until all possible coin combinations are explored.\\
We can keep track of the shortest path through the resulting tree which has a leaf value of 0, to avoid storing the entire tree in memory.\\
We return the length of the shortest path as $r$.\\


% TODO NOTE: PLOT OF ALGORITHM HERE

A sample python implementation is shown in figure \ref{fig:coin-change-bf}.

\begin{figure}[h]
    \centering
    \begin{lstlisting}
    def coin_change_bf(D, a):
        def dfs(a):
            if a == 0:
                return 0
            if a<0:
                return float('inf')
            return min([1+dfs(a-c) for c in D])
        minimum = dfs(a)
        return minimum if minimum < float("inf") else -1
    \end{lstlisting}
    \caption{Coin Change Brute Force Python Implementation}
    \label{fig:coin-change-bf}
\end{figure}

\subsection{Complexity Analysis of the Brute Force Approach to the Coin Change Problem}

\begin{description}
    \item[Time Complexity:]
    For the worst case scenario, let's assume each coin denomination $c \in D < a$ such that each node which is not a leaf node has $|D|$ children. This means we have $|D|$ recursive calls at the first level, $|D|^{2}$ at the second level, $|D|^{n}$ at the $n$'th level.\\

    The total number of recursive calls in this scenario is $|D| + |D|^{2} + ... + |D|^{a}$ which is $O(|D|^a)$.
    
    Therefore the time complexity is $O(|D|^{a})$. This is because at each step, there are $|D|$ choices (coin denominations) to consider, and the recursion depth is at most $a$.
    
        
    \item[Space Complexity:] 
        We do not store the entire tree in memory, only the current path.

        The space complexity is determined by the maximum depth of the recursion stack. In the worst case, the recursion depth is equal to the target amount $a$. Therefore, the space complexity is $O(a)$.
        
        
    \item[Overall:] total\\
        Time Complexity: $O(|D|^a)$

        Space Complexity: $O(a)$
        
\end{description}

\subsection{Memoization Approach to the Coin Change Problem}

In the brute force algorithm, we have a chance to arrive at a value multiple times. For every path in the search tree,
we can store intermediate results in a table which we will call $memo$ (for memoization) 
so that the next time we arrive at a value, eg. 3, we don't have to repeat the work in finding the minimum amount of extra coins needed to sum to $a$.
Instead we can simply look in the table with a constant time lookup.\\
This optimization reduces search time greatly, as seen in subsection \ref{subsec:ca-coin-change-memo}.\\
A sample python implementation is shown in figure \ref{fig:coin-change-memo}.

\begin{figure}[H]
    \centering
    \begin{lstlisting}
    def coin_change_memo(D, a):
    memo = {}
    def dfs(a):
        if a == 0:
            return 0
        if a < 0:
            return float('inf')
        if a in memo:
            return memo[a]
        
        memo[a] = min([1+dfs(a-c) for c in D])
        return memo[a]
            
    res = dfs(a)
    return res if res < float("inf") else -1
    \end{lstlisting}
    \caption{Coin Change Memoization Python Implementation}
    \label{fig:coin-change-memo}
\end{figure}

\subsection{Complexity Analysis of the Memoization Approach to the Coin Change Problem}\label{subsec:ca-coin-change-memo}

\begin{description}
    \item[Time Complexity:]
    %   CITATION HERE TO THE PYTHON MANUAL
        Each unique subproblem is evaluated once, and the next time it is encountered it is retrieved from the memoization table with a constant time lookup\footnote{Python dictionary lookups have an expected $O(1)*$ time complexity.}.
        As there are $|D| * a$ unique subproblems in the worst case\footnote{$|D|$ constant time subtractions from any intermediate value $v$ where $0 \leq v \leq a$.}, the time complexity to solve all of them is $O(|D| * a)$.
    
        
    \item[Space Complexity:] 
        We need to store the $memo$ table in memory.
        The memoization table is represented by a lookup data structure where the keys range from 0 to a,
        representing the solution to each unique subproblem.
        Hence, the memory required to store the table is of order $O(a)$.
        
    \item[Overall:] total\\
        Time Complexity: $O(|D| * a)$

        Space Complexity: $O(a)$
        
\end{description}

\subsection{Tabulation Approach to the Coin Change Problem}

Instead of doing a dfs to fill in the memo table, which requires a traversal of the exponential search tree, we can calculate the values in the memo table directly, and extract the answer from there.\\
We will call the $memo$ table $dp$, as we are no longer doing memoization, but tabulation.\\
$dp[i]$ represents the minimum amount of coins needed to get the amount $i$.\\
For the example $D = [5,4,3,1]$, $a = 7$\\
We initialize each $dp[i]$ to contain infinity.\\
We know that $dp[0] = 0$ as it takes $0$ coins to add up to an amount of $0$. We can initialize this in our table.\\
Now we can deduce $dp[1], dp[2], ... dp[a]$. $dp[a]$ will contain $r$.\\
To get $dp[i]$, we will look at each coin $c \in D$ in sequence.\\
For each $c \in D$, we take $i - c$ to get $t$, and look for $dp[t]$ if it exists.\\
Our intermediate result is $1 + dp[t]$\\
If this result is less than the current $dp[i]$ and is not negative, we update $dp[i] \leftarrow 1 + dp[t]$.\\
The logic of this is that the amount of coins it takes to make the amount $dp[i]$ is the amount of coins it takes to make the amount $dp[t]$ plus one.\\
The logic is demonstrated with the examples:\\
Example 1: Calculating $dp[1]$\\
$dp[0]=0$\\
$dp[1]=\infty$\\
$dp[2]=\infty$\\
$dp[3]=\infty$\\
$dp[4]=\infty$\\
$dp[5]=\infty$\\
$dp[6]=\infty$\\
To calculate $dp[1]$:\\
For $c \in D = [5,4,3,1]$\\
$t = i-c = 1-5 = -4$, ignore because negative\\
$t = i-c = 1-4 = -3$, ignore because negative\\
$t = i-c = 1-3 = -2$, ignore because negative\\
$t = i-c = 1-1 = 0$\\
Look up the value of $dp[t] = 0$.\\
Now we take $1 + dp[0] = 1$.\\
This means a possible solution to $dp[1]$ is 1.\\
Since $1 < \infty$, we update $dp[1] \rightarrow 1$\\
Example 2: Calculating $dp[7]$\\
$dp[0]=0$\\
$dp[1]=1$\\
$dp[2]=2$\\
$dp[3]=1$\\
$dp[4]=1$\\
$dp[5]=1$\\
$dp[6]=2$\\
$dp[7]=\infty$\\
To calculate $dp[7]$:\\
For each $c \in D = [5,4,3,1]$:\\
$t = i-c = 7-5 = 2$, $dp[2] = 2$, $1 + dp[2] = 3$, $3 < \infty$, update $dp[7] \rightarrow 3$\\
$t = i-c = 7-4 = 3$, $dp[3] = 1$, $1+dp[3] = 2$, $2 < 3$, update $dp[7] \rightarrow 2$\\
$t = i-c = 7-3 = 4$, $dp[4] = 1$, $1+dp[4] = 2$, $2 = 2$, ignore\\
$t = i-c = 7-1 = 6$, $dp[6] = 2$, $1+dp[6] = 3$, $3 > 2$, ignore\\
We conclude that the minimum solution to $dp[7] is 2$, achieved by adding a 4 coin to $dp[3]$, which is achieved by adding a 3 coin to $dp[0]$\\
A sample python implementation is shown in figure \ref{fig:coin-change-dp}.\\

\begin{figure}[H]
    \centering
    \begin{lstlisting}
    def coin_change_dp(D,a):
        dp=[float('inf')] * (a + 1)
        dp[0] = 0
    
        for i in range(1, a+1):
            for c in D:
                t = i - c
                if t >= 0:
                    dp[i] = min(dp[i], 1+dp[t])
    
        return dp[a] if dp[a] != float('inf') else -1
    \end{lstlisting}
    \caption{Coin Change Tabulation Python Implementation}
    \label{fig:coin-change-dp}
\end{figure}

\subsection{Complexity Analysis of the Tabulation Approach to the Coin Change Problem}

\begin{description}
    \item[Time Complexity:]
        % REFERENCE TO PYTHON MANUAL
        For the worst case scenario, we need to iterate for all $i = 0; i <= a; i++$.
        And for each $i$, we need to iterate over each coin $c \in D$.
        All other operations within the loops are constant time lookups and subtractions, so the time complexity is $O(|D| * a)$
            
    \item[Space Complexity:] 
        The space complexity is determined by the size of the $dp$ array. This array is always of size $a+1$.
        Therefore the space complexity is $O(a)$
        
    \item[Overall:] Total:\\
        Time Complexity: $O(|D| * a)$

        Space Complexity: $O(a)$
        
\end{description}