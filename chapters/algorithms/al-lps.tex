\begin{description}
    \item[Problem Statement:]
        Given a string, return the length of the longest subsequence of the string which is a palindrome. A palindrome is a string which is identical to itself when reversed (Example: "racecar").
    
    \item[Input:]
        A string $s$.
        
    \item[Output:]
        An integer $lps$ which represents the longest palondromic subsequence of $s$.
        
    \item[Example:] For:\\
        $s = "babbb"$\\
        $lps = 4$
        
    \item[Explanation:]
        $"bbbb"$ is the longest palindromic subsequence of $s$. It has length $4$.
\end{description}

\subsection{Tabulation Approach to Longest Palindromic Subsequence}
This problem is simply a special case of Longest Common Subsequence,
which is covered in Section \ref{section:lcs}.
The longest palindromic subsequence of a string $s$ is equal to the longest common subsequence of $s$ and $reverse(s)$.
Therefore the logic of Longest Common Subsequence can be used to solve of Longest Palindromic Subseqence, if we make $text1=s$ and $text2=reverse(s)$.\\
For the implementation of the $lcs$ subroutine, see Figure \ref{fig:lcs-dp}.

\subsection{Complexity Analysis of Longest Palindromic Subsequence}
Let $n$ be the length of $s$.
\begin{description}
    \item[Time Complexity:]
        The time complexity of reversing $s$ is $O(n)$.
        The time complexity of finding the longest common subsequence of $s$ and $reverse(s)$ is $O(n^2)$, as proved in Section \ref{subsec:ca-lcs-dp}.
        In total this is of order $O(n^2)$.
        
    \item[Space Complexity:] 
        The space complexity of finding the longest common subsequence of $s$ and $reverse(s)$ is $O(n^2)$ as proved in Section \ref{subsec:ca-lcs-dp},
        but can be optimized to $O(n)$ using the two row method described in Section \ref{subsec:lcs-optimized}.
        
    \item[Overall:] Total:\\
        Time Complexity: $O(n^2)$\\
        Space Complexity: $O(n)$
    
\end{description}