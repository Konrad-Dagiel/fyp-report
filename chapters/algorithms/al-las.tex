\begin{description}
    \item[Problem Statement:]
        Given an array, find the length of the longest alternating subsequence in the array.
        A sequence ${x1,x2,x3,x4...xn}$ is alternating if its elements satisfy one of the following:

        $x1>x2<x3>x4<...$
        
        $x1<x2>x3<x4>...$
        
    \item[Input:]
        An integer array $nums$.
        
    \item[Output:] 
        An integer $max\_length$.
        
    \item[Example:] For:\\
        $nums = [1,17,5,10,13,15,10,5,16,8]$\\
        $max_length = 7$

    \item[Explanation:]
        The longest alternating subsequence in $nums$ is $[1,17,5,15,5,16,8]$ which has length $7$.

\end{description}

    

\subsection{Brute Force Approach to Longest Alternating Subsequence}

The brute force way to approach this problem is to generate every possible subsequence of $nums$. 
For each subsequence, we can check if it is alternating iteratively, and if it is, calculate its length.
We keep track of the overall maximum length as we iterate through the subsequences. This way we are able to store just one subsequence at a time.

An implementation of the brute force algorithm is given in Figure \ref{fig:las-bf}.

\begin{figure}[H]
    \centering
    \begin{lstlisting}
    def is_alternating(sequence):
        if len(sequence) < 3:
            return True
    
        for i in range(1, len(sequence) - 1):
            if not ((sequence[i - 1] > sequence[i] < sequence[i + 1]) or
                    (sequence[i - 1] < sequence[i] > sequence[i + 1])):
                return False
        return True
    
    def las_bf(nums):
        if not nums:
            return 0
    
        n = len(nums)
        max_length = 1
    
        for i in range(1 << n):
            subsequence = [nums[j] for j in range(n) if (i & (1 << j)) > 0]
            if is_alternating(subsequence):
                max_length = max(max_length, len(subsequence))
    
        return max_length
    \end{lstlisting}
    \caption{Longest Alternating Subsequence Brute Force Python Implementation}
    \label{fig:las-bf}
\end{figure}


NOTE: In line 18 in Figure \ref{fig:las-bf}, The expression $1 << n$ represents a bitwise left shift operation.
In Python, $<<$ is the left shift operator, and it shifts the binary representation of the number to the left by $n$ positions.
In the context of generating all possible subsequences, $1 << n$ is used to create a bitmask with the rightmost $n$ bits set to $1$.
Each bit in the bitmask corresponds to whether the corresponding element in the array is included or excluded in the current subsequence.
The line "for $i$ in range$(1 << n)$" generates all possible subsequences by iterating through all bitmasks from 0 to $2^n-1$.

\subsection{Complexity Analysis of the Brute Force Approach to Longest Alternating Subsequence}
Let n be the length of nums.
\begin{description}
    \item[Time Complexity:]
        The complexity of generating every possible subsequence is $O(2^n)$.
        Checking if a sequence is alternating is $O(|sequence|)$, which in the worst case is $n$,
        and getting the length of a subsequence, as well as comparing it to the $max\_length$ are $O(1)$.
        This is overall of order $O(2^n)$.
        
    \item[Space Complexity:] 
        The space complexity is $O(n)$, as we must store a single subsequence at a time,
        which in the worst case is of length $n$.
        
    \item[Overall:] Total:\\
        Time Complexity: $O(2^n)$\\
        Space Complexity: $O(n)$
    
\end{description}


\subsection{Auxiliary Arrays Solution for Longest Alternating Subsequence}
Notice that for an aribitrary longest alternating subsequence of length $k$ ending at index $i$,
if there exists exactly one number which can extend the sequence,
the length of the total longest alternating subsequence is guaranteed to be $k+1$.
This shows the optimal substructure property.
Also, when looking for subsequences at index $i$,
we must re-compute all of the subsequences at index $i-1$.
This shows the overlapping subproblems property.

As using memoization on subsequence problems is not space efficient,
we can use tabulation and only store the necessary information rather than an intermediate result for every subsequence.
The following tabulation solution to Longest Alternating Subsequence is commonly referred to as the Auxiliary Arrays solution.

Two auxiliary arrays $inc$ and $dec$, of length $|nums|$ are initialized.
$inc[i]$ contains the length of the longest alternating subsequence of $nums[0:i]$,
where the last element of the subsequence is greater than the previous element.
$dec[i]$ contains the length of the longest alternating subarray of $nums[0:i]$,
where the last element of the subsequence is less than the previous element.\\

The algorithm iterates through $nums$, considering each element $nums[i]$ and updating the inc and dec arrays based on the following conditions:
\newpage

\begin{enumerate}
    \item If $nums[i]$ is greater than $nums[j]$ for some previous index $j$, it means the sequence can be extended in an increasing manner.
    In this case, $inc[i]$ is updated to be the maximum of its current value and the length of the longest decreasing subsequence ending at index $j+1$, found in $dec[j+1]$.

    \item If $nums[i]$ is smaller than $nums[j]$ for some previous index $j$, it means the sequence can be extended in a decreasing manner.
    In this case, $dec[i]$ is updated to be the maximum of its current value and the length of the longest increasing subsequence ending at index $j+1$, found in $inc[j+1]$.

    \item The length of the longest alternating subsequence is the maximum value in the $inc$ and $dec$ arrays.
\end{enumerate}


A sample python implementation is shown in Figure \ref{fig:las-auxiliary}.

\begin{figure}[H]
    \centering
    \begin{lstlisting}
    def las_auxiliary(nums):
        n = len(nums)
        
        inc = [1] * n
        dec = [1] * n
    
        for i in range(1, n):
            for j in range(i):
                if nums[i] > nums[j]:
                    inc[i] = max(inc[i], dec[j] + 1)
                elif nums[i] < nums[j]:
                    dec[i] = max(dec[i], inc[j] + 1)

        return max(max(inc), max(dec))
    \end{lstlisting}
    \caption{Longest Alternating Subsequence Auxiliary Arrays Python Implementation}
    \label{fig:las-auxiliary}
\end{figure}
\newpage

\subsection{Complexity Analysis of the Auxiliary Arrays Approach to Longest Alternating Subsequence}
Let n be the length of nums.
\begin{description}
    \item[Time Complexity:]
        We use a double nested for loop to iterate over $nums$.
        This gives a complexity of $O(n^2)$.
        All other operations are either constant time lookups, updates or $max(a,b)$,
        all of which are $O(1)$.
        
    \item[Space Complexity:] 
        The space complexity is $O(2n)$, which is of order $O(n)$.
        This is because we must store the $inc$ array and the $dec$ array,
        each of which have the same length as $nums$.
        All other variables are stored with constant space complexity.

        
    \item[Overall:] Total:\\
        Time Complexity: $O(n^2)$\\
        Space Complexity: $O(n)$
    
\end{description}


\subsection{Optimized Soluiton to Longest Alternating Subsequence}
Observe that at each step, the max of $inc$ and $dec$ can be found at the current index of the arrays,
so we do not need to store the entire $inc$ and $dec$ arrays.
Instead, we can use an $inc$ and $dec$ variable which will hold the value stored at $inc[i]$ and $dec[i]$ respectively.
This is because a maximum alternating subsqeuence of $nums[0:a]$ will never be of lower length than a maximum alternating subsequence of $nums[0:b]$ if $a > b$.

We can do a single iteration over nums where:

\begin{enumerate}
    \item $inc$ should be set to $dec + 1$, if and only if the last element in the alternating sequence was less than its previous element.
    
    \item $dec$ should be set to $inc + 1$, if and only if the last element in the alternating sequence was greater than its previous element.

\end{enumerate}

A sample python implementation is shown in Figure \ref{fig:las-optimized}.

\begin{figure}[H]
    \centering
    \begin{lstlisting}
    def las_optimized(nums):
        n = len(nums)
    
        inc = 1
        dec = 1
    
        for i in range(1, n):
            if nums[i] > nums[i - 1]:
                inc = dec + 1
            elif nums[i] < nums[i - 1]:
                dec = inc + 1
    
        result = max(inc, dec)
        return result
    \end{lstlisting}
    \caption{Longest Alternating Subsequence Optimzied Python Implementation}
    \label{fig:las-optimized}
\end{figure}

\subsection{Complexity Analysis of the Optimized Approach to Longest Alternating Subsequence}
Let n be the length of nums.

\begin{description}
    \item[Time Complexity:]
        We use a single for loop to iterate over $nums$.
        This gives a complexity of $O(n)$.
        All other operations are either constant time lookups,
        updates or $max(a,b)$, all of which are $O(1)$.
        
    \item[Space Complexity:] 
        The space complexity is $O(1)$ as we only need to store a single value for $inc$ and $dec$.

    \item[Overall:] Total:\\
        Time Complexity: $O(n)$\\
        Space Complexity: $O(1)$
    
\end{description}

\newpage