This famous dynamic programming algorithm is used for local alignment of DNA and protein sequences.

\begin{description}
    
    \item[Problem Statement:]
        Given two character sequences $seq1$ and $seq2$,
        remove characters from the beginning or end of $seq1$ and/or $seq2$ such that the maximum number of characters in $seq1$ and $seq2$ are 'aligned'.
        Characters $X$ and $Y$ are 'aligned' when $X == Y$ and the index of $X$ in $seq1$ is exactly equal to the index of $Y$ in $seq2$.
        
    \item[Input:] 
        Two strings $seq1$ and $seq2$.
        
    \item[Output:]
        Two strings $align1$ and $align2$.
        
    \item[Example:] For:\\
        $seq1 = "ATGCT"$\\
        $seq2 = "AGCT"$\\
        $align1 = "GCT"$\\
        $align2 = "GCT"$

    \item[Explanation:]
        Locally Aligned sequences:\\
        G C T\\
        G C T
\end{description}

The trick is to find an efficient way to trim the sequences, such that the number of aligned letters is maximised. In this case it is 3.

This is very similar to Needleman-Wunsch. The algorithm is the same as above, however all negative values become zero as follows:

Create a matrix of size: $(len(seq1) + 1)$ x $(len(seq2) + 1)$ as follows:

\begin{table}[htbp]
    \centering
    \begin{tabular}{|c|c|c|c|c|c|c|}
        \hline
          &  & \textbf{A} & \textbf{T} & \textbf{G} & \textbf{C} & \textbf{T} \\
        \hline
         &  &  &  &  &  &  \\
        \hline
        \textbf{A} &  &  &  &  &  &  \\
        \hline
        \textbf{G} &  &  &  &  &  &  \\
        \hline
        \textbf{C} &  &  &  &  &  &  \\
        \hline
        \textbf{T} &  &  &  &  &  &  \\
        \hline
    \end{tabular}
\end{table}


Now use the following scheme to fill in the table:

$Match: 1$

$Mismatch: -1$

$GAP: -2$

\subsection{Initialization of the Smith-Waterman Table}

Starting with 0 at $(0,0)$, fill the extra row and column with progressive GAP penalties as follows [In the Smith-Waterman Algorithm, all negative numbers become 0]:

\begin{table}[htbp]
    \centering
    \begin{tabular}{|c|c|c|c|c|c|c|}
        \hline
          &  & \textbf{A} & \textbf{T} & \textbf{G} & \textbf{C} & \textbf{T} \\
        \hline
        & 0 & 0 & 0 & 0 & 0 & 0 \\
        \hline
        \textbf{A} & 0 &  &  &  &  &  \\
        \hline
        \textbf{G} & 0 &  &  &  &  &  \\
        \hline
        \textbf{C} & 0 &  &  &  &  &  \\
        \hline
        \textbf{T} & 0 &  &  &  &  &  \\
        \hline
    \end{tabular}
\end{table}

\subsection{Filling the Smith-Waterman Table}

Now starting at $(1,1)$, and going row-wise, fill each cell with the max of the following, converting any negative values to 0:

\begin{enumerate}
    \item Value from left + $GAP$
    
    \item Value from above + $GAP$
    
    \item Value from diagonal + ($Match$ or $Mismatch$) [depending on wether the row and column label are the same letter]

\end{enumerate}

So, for the $(1,1)$ cell, we fill it with $max(-4,-4,1) = 1$.

And for the $(1,2)$ cell, we fill it with $max(-1,-6,-3) = -1 \rightarrow 0$.

And so on...

Until we get:

\begin{table}[htbp]
    \centering
    \begin{tabular}{|c|c|c|c|c|c|c|}
        \hline
          &  & \textbf{A} & \textbf{T} & \textbf{G} & \textbf{C} & \textbf{T} \\
        \hline
        & 0 & 0 & 0 & 0 & 0 & 0 \\
        \hline
        \textbf{A} & 0 & 1 & 0 & 0 & 0 & 0 \\
        \hline
        \textbf{G} & 0 & 0 & 0 & 1 & 0 & 0 \\
        \hline
        \textbf{C} & 0 & 0 & 0 & 0 & 2 & 0 \\
        \hline
        \textbf{T} & 0 & 0 & 1 & 0 & 0 & 3 \\
        \hline
    \end{tabular}
\end{table}

\subsection{Traceback in the Smith-Waterman Table}

Starting from the highest value in the matrix, move diagonally backwards until any 0 is reached.

For each diagonal movement, align the coresponding characters.

Note, the characters are aligned in reverse.

A sample python implementation is given in Figure \ref{fig:smith-waterman}

\begin{figure}[H]
    \centering
    \begin{lstlisting}
    def smith_waterman(seq1, seq2, match=1, mismatch=-1, gap=-2):
        m, n = len(seq2), len(seq1)
        score = [[0] * (n + 1) for _ in range(m + 1)]
    
        # Initialize the first row and column
        for i in range(m + 1):
            score[i][0] = 0
        for j in range(n + 1):
            score[0][j] = 0
    
        # Fill in the scoring matrix
        max_score = 0
        max_position = (0, 0)
    
        for i in range(1, m + 1):
            for j in range(1, n + 1):
                match_mismatch = match if seq2[i - 1] == seq1[j - 1] else mismatch
                diagonal = score[i - 1][j - 1] + match_mismatch
                horizontal = score[i][j - 1] + gap
                vertical = score[i - 1][j] + gap
    
                score[i][j] = max(0, diagonal, horizontal, vertical)
    
                if score[i][j] > max_score:
                    max_score = score[i][j]
                    max_position = (i, j)
    
        # Traceback to find the alignment
        align1, align2 = "", ""
        i, j = max_position
    
        while i > 0 and j > 0 and score[i][j] > 0:
            if score[i][j] == score[i - 1][j - 1] + (match if seq2[i - 1] == seq1[j - 1] else mismatch):
                align2 = seq2[i - 1] + align2
                align1 = seq1[j - 1] + align1
                i -= 1
                j -= 1
            elif score[i][j] == score[i][j - 1] + gap:
                align2 = seq2[i - 1] + align2
                align1 = "-" + align1
                j -= 1
            else:
                align2 = "-" + align2
                align1 = seq1[j - 1] + align1
                i -= 1
    
        return align1, align2
    
    sequence1 = "ATGCT"
    sequence2 = "AGCT"
    alignment1, alignment2 = smith_waterman(sequence1, sequence2)
    \end{lstlisting}
    \caption{The Needleman Wunsch Algorithm Python Implementation}
    \label{fig:smith-waterman}
\end{figure}

