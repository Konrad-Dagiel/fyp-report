\begin{description}
    \item[Problem Statement:]
        Given as input positive integers $n$ and $k$ where $n \geq k$, calculate how many different ways you can select $k$ unique examples from a set of size $n$.
        In other words, calculate $C(n,k)$ where: $$C(n,k) = n! / k! ((n-k)!)$$

    \item[Input:] 
        Two positive integers $C(n,k)$
        
    \item[Output:]
        A single positive integer $C(n,k)$
        
    \item[Example:] For:\\
        $n = 4$\\
        $k = 2$\\
        $C(n,k) = 6$

    \item[Explanation:]
        There are 6 unique ways you can choose 2 different examples from a set of size 4.
        
\end{description}

We know that by definition:

$$C(n,k) = C(n-1,k-1) + C(n-1,k)$$

$$C(n,0) = 1$$

$$C(n,n) = 1$$

We can use this as a recursive case and base cases in a brute force solution.

\subsection{Brute Force Approach to Binomial Coefficients}
We can use recursion to arrive at our answer, using the definition as our recursive case and our base cases.

A sample python implementation is shown in Figure \ref{fig:binomial-bf}.

\begin{figure}[H]
    \centering
    \begin{lstlisting}
    def C_bf(n,k):
        if k == 0 or k == n: return 1
        return C_bf(n-1,k-1) + C_bf(n-1,k)
    \end{lstlisting}
    \caption{Binomial Coefficients Brute Force Python Implementation}
    \label{fig:binomial-bf}
\end{figure}

\subsection{Complexity Analysis of the Brute Force Approach to Binomial Coefficients}
\begin{description}
    \item[Time Complexity:]
        For our analysis, in the worst case $k$ is equal to $n/2$.
        This means that at each recursion,
        we create two subproblems.
        We do this $n$ times, giving a total time complexity of $O(2^n)$.

    \item[Space Complexity:] 
        The space complexity is determined by the depth of recursion, which is $O(n)$.

    \item[Overall:] Total:\\
        Time Complexity: $O(2^n)$\\
        Space Complexity: $O(n)$
    
\end{description}

\subsection{Memoization Approach to Binomial Coefficients}
If we look at the trace of calculating $C(4,2)$ in our recursive algorithm, we can see that it is guaranteed to be the sum of $C(3,1)$ and $C(3,2)$, as by definition $C(n,k) = C(n-1,k-1) + C(n-1, k)$.
This shows the problem has the optimal substructure property.
To calculate $C(3,1)$ we must know $C(2,1)$.
This is also true when calculating $C(3,2)$.
This means there is repeated calculations of $C(2,1)$,
showing the overlapping subproblems property.
In larger problems, the number of repeated calculations is vast.
We can use memoization to store each calculation we do in a table called $memo$ to avoid repeating subproblems.
At each step, before continuing with recursion, we check $memo$ to see if the calculation has been done already, and if so, we take the value from the $memo$ table instead of making a recursive call.

A sample python implementation is shown in Figure \ref{fig:binomial-memo}.

\begin{figure}[H]
    \centering
    \begin{lstlisting}
    def C_memo(n,k,memo={}):
        if k == 0 or k == n: return 1
    
        if (n,k) in memo:
            return memo[(n,k)]
        
        result = C_memo(n-1,k-1,memo) + C_memo(n-1,k,memo)
        memo[(n,k)] = result
        return result
    \end{lstlisting}
    \caption{Binomial Coefficeints Memoization Python Implementation}
    \label{fig:binomial-memo}
\end{figure}

\subsection{Complexity Analysis of the Memoization Approach to Binomial Coefficients}
\begin{description}
    \item[Time Complexity:]
        The memoization table ensures that each unique subproblem is solved only once.
        In the case of "$n$ choose $k$," there are $O(n * k)$ unique subproblems because the parameters $n$ and $k$ can take values from $0$ to $n$.
        The rest of the table lookups are expected constant time as we are using a dictionary for the $memo$ table.

    \item[Space Complexity:] 
        We must store the $memo$ table, which is a dictionary of size $O(n * k)$ as $n$ and $k$ can take values from $0$ to $n$.

    \item[Overall:] Total:\\
        Time Complexity: $O(n * k)$\\
        Space Complexity: $O(n * k)$
    
\end{description}

\subsection{Tabulation Approach to Binomial Coefficients}
Instead of recursively creating the table by searching the entire tree of possible $n$ and $k$ values, we can use tabulation to fill in a table from which we can extract our answer.
Consider a table $dp$ with dimensions ($n+1$) x ($k+1$), where $dp[i][j]$ in the table contains the result of $C(i,j)$
We could build this table up starting from our base cases:
$$C(n,0) = 1$$

$$C(n,n) = 1$$

, until we have a solution to $C(n,k)$.

So for the calculation of $C(4,2)$ for example, we would initialize the following table:

\begin{table}[htbp]
    \centering
    \begin{tabular}{|c|c|c|c|}
        \hline
          & \textbf{0} & \textbf{1} & \textbf{2} \\
        \hline
        \textbf{0} & 1 & 0 & 0 \\
        \hline
        \textbf{1} & 1 & 1 & 0 \\
        \hline
        \textbf{2} & 1 &   & 1 \\
        \hline
        \textbf{3} & 1 &   &  \\
        \hline
        \textbf{4} & 1 &   &   \\
        \hline
    \end{tabular}
\end{table}
Now, we can use $C(n,k) = C(n-1,k-1) + C(n-1,k)$ to fill the table.

For example, to get $dp[2][1]$ $(C(2,1))$, we take $dp[1][0] + dp[1][1] = 2$.

Do this for the entire table as follows:

\begin{table}[htbp]
    \centering
    \begin{tabular}{|c|c|c|c|}
        \hline
          & \textbf{0} & \textbf{1} & \textbf{2} \\
        \hline
        \textbf{0} & 1 & 0 & 0 \\
        \hline
        \textbf{1} & 1 & 1 & 0 \\
        \hline
        \textbf{2} & 1 & 2 & 1 \\
        \hline
        \textbf{3} & 1 & 3 & 3 \\
        \hline
        \textbf{4} & 1 & 4 & 6 \\
        \hline
    \end{tabular}
\end{table}

Until we arrive at $dp[n][k]$ which will give the answer of $C(n,k)$.

A sample python implementation is shown in Figure \ref{fig:binomial-dp}.

\begin{figure}[H]
    \centering
    \begin{lstlisting}
    def C_dp(n,k):
        dp = [[0] * (k+1) for _ in range(n+1)]
    
        #Fill in the base cases
        for i in range(n+1):
            dp[i][0] = 1
            dp[i][min(i,k)]=1
    
        #Fill in the rest of the table using the definition of C(n,k)
        for i in range(1,n+1):
            for j in range(1,min(i,k)+1):
                dp[i][j] = dp[i-1][j-1] + dp[i-1][j]
                
        return dp[n][k]
    \end{lstlisting}
    \caption{Binomial Coefficients Tabulation Python Implementation}
    \label{fig:binomial-dp}
\end{figure}

We do not fill the table where $k > n$.
This because it is impossible to choose $k$ unique examples from a set of size $n$.
This is done by only iterating to $min(i,k)$ in the inner loop.

\subsection{Complexity Analysis of the Tabulation Approach to Binomial Coefficients}

\begin{description}
    \item[Time Complexity:]
        We have to fill in a table of size $O(n * k)$ with values, and each value is obtained through a constant time lookup and addition.

    \item[Space Complexity:] 
        We must store the $dp$ table, which is a 2D array of size $O(n * k)$ as $n$ and $k$ can take values from $0$ to $n$.
        
    \item[Overall:] Total:\\
        Time Complexity: $O(n * k)$\\
        Space Complexity: $O(n * k)$
    
\end{description}

\subsection{Optimzied Tabulation Approach to Binomial Coefficeints}
Notice that we can calculate the values in any row $r$ using the values in row $r-1$.
Hence, we do not need to store the entire $dp$ table in memory, only two rows at a time.
The current row $r$, and the previous row $r-1$.
This reduces space complexity from $O(n * k)$ to $O(k)$.
Notice also that $C(n,k) = C(n,n-k)$.
We can therefore set $k = n-k$ if $k > n-k$ and $n-k$ is positive before we start our algorithm, and our result will be the same.
This reduces the time complexity from $O(n * k)$ to $O(n * min(k,n-k))$.

A sample python implementation is shown in Figure \ref{fig:binomial-optimized}.

\begin{figure}[H]
    \centering
    \begin{lstlisting}
    def C_optimized(n,k):
        if k > n-k and n-k >= 0:
            k = n-k
        oldRow = [0] * (k+1)
        oldRow[0] = 1
    
        for i in range(1,n+1):
            newRow = [0] * (k+1)
            newRow[0] = 1
            for j in range(1,min(i,k)+1):
                newRow[j] = oldRow[j-1] + oldRow[j]
    
            oldRow = newRow
              
        return newRow[k]
    \end{lstlisting}
    \caption{Binomial Coefficients Optimized Python Implementation}
    \label{fig:binomial-optimized}
\end{figure}

