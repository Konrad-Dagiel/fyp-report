
\begin{description}
    \item[Problem Statement:]
        Given an array, return the length of the longest strictly increasing subsequence in the array.
        A sequence is said to be increasing if and only if $x1<x2<x3<...xn$.
        A subsequence does not have to be contiguous.

    \item[Input:]
        An integer array $nums$.
        
    \item[Output:]
        An integer $lis$, the length of the longest increasing subsequence of $nums$.
    \item[Example:] For:\\
        $nums = [2,5,3,7,101,18]$\\
        $lis = 4$

    \item[Explanation:]
    The subsequence $[2,5,7,101]$ is the longest increasing subsequence in $nums$, and has length $4$.

        
\end{description}

Much like coin change, this problem appears trivial at first glance. One may attempt to be greedy as follows:

\subsection{Greedy Approach to Longest Increasing Subsequence}

\begin{algorithm}
    \caption{Greedy Approach to Longest Increasing Subsequence}
    \label{algo:lis-greedy}
    \KwIn{An integer array $nums$.}
    \KwOut{An integer $lis$, the longest increasing subsequence in $nums$.}
    $lis := 0$\\
    $index := 0$\\
    $cur := nums[0]$\\
    \While{$index \leq |nums|$}{
        \If{$nums[index] > cur$}{
            $lis += 1$\\
            $cur := nums[index]$
        }
        $index +=1$
    }
    \KwRet{$lis$}
\end{algorithm}


In Algorithm \ref{algo:lis-greedy} we iterate through nums keeping track of the current max value enountered, incrementing our result each time a new larger value is encountered.

\subsection{Optimality of the Greedy Approach to Longest Increasing Subsequence}

This algorithm is not optimal however, and we can prove this by counter-example.
Take:$$nums = [10,9,2,5,3,7,101,18]$$
Given these inputs, the greedy result is: $r1 = 2  ([10,101])$.\\
An optimal solution for these inputs is: $r2 = 4 ([2,3,7,101])$.\\
We see that $r1 > r2$, meaning the greedy approach does not find the maximised solution.

We therefore need a more sophisticated approach.

\subsection{Brute Force Approach to Longest Increasing Subsequence}

We can try a brute force approach, where we start at $nums[0]$, and for each $i \in nums$ generate two subsequences,
one where we  exclude it from the subsequence and one where we include it in the subsequence if
$nums[current\_index] > nums[prev\_index]$ (So that we ensure each subsequence generated is strictly increasing).
We keep track of the $prev\_index$, which represents the last index we included in the result,
and the $current\_index$, which is the index for which we are making the choice.
This will generate all possible increasing subsequenes.
We keep track of the length of the longest increasing subsequence, and return it once all increasing subsequences are explored.

A sample python implementation is shown in Figure \ref{fig:lis-bf}.

\begin{figure}[H]
    \centering
    \begin{lstlisting}
    def lis_bf(nums):
        def dfs(prev_index, current_index):
            # Base case: reached the end of the sequence
            if current_index == len(nums):
                return 0
    
            # Case 1: Exclude the current element
            exclude_current = dfs(prev_index, current_index + 1)
    
            # Case 2: Include the current element if it is greater than the previous one
            include_current = 0
            if prev_index < 0 or nums[current_index] > nums[prev_index]:
                include_current = 1 + dfs(current_index, current_index + 1)
    
            # Return the maximum length of the two cases
            return max(exclude_current, include_current)
    
        # Start the recursion with initial indices (-1 represents no previous index)
        return dfs(-1, 0)
    \end{lstlisting}
    \caption{Longest Increasing Subsequence Brute Force Python Implementation}
    \label{fig:lis-bf}
\end{figure}

\subsection{Complexity Analysis of the Brute Force Approach to Longest Increasing Subsequence}
Let $n$ be the length of $nums$.
\begin{description}
    \item[Time Complexity:]
        For the worst case scenario, There are $n$ indices to consider.
        There are two subtrees at each decision, one where we include the current index, and one where we do not.
        This brings the time complexity to $O(2^n)$.
        
    \item[Space Complexity:] 
        The space complexity is determined by the recursion depth,
        as once we explore a path in the recursion tree, we can release it from memory when we go to the next path.
        Therefore the space complexity is $O(n)$.
        
        
    \item[Overall:] Total:\\
        Time Complexity: $O(2^n)$\\
        Space Complexity: $O(n)$
        

\end{description}


\subsection{Memoization Approach to Longest Increasing Subsequence}

Notice that for an aribitrary longest increasing subsequence of length $k$ ending at index $i$,
if there exists exactly one number which can extend the sequence, the length of the total subsequence is guaranteed to be $k+1$.
This shows the optimal substructure property.
Also, when looking for subsequences at index $i$,
we must re-compute all of the subsequences at index $i-1$.
This shows the overlapping subproblems property.
We can use memoization to avoid repeating subproblems, such as when we are deciding wether the next element should be added or not for multiple subsequences ending in the same element.
The memoization approach goes as follows:
We initialize an empty dictionary called $memo$,
the keys of which are constructed by making a tuple of $(prev\_index, current\_index)$.
Before proceeding with the recursive calls, the function checks if the result for the current combination of $prev\_index$ and $current\_index$ is already computed and stored in the $memo$ dictionary.
If it is, the stored result is returned immediately.
This optimization allows the algorithm to avoid repeating work,
speeding up the runtime significantly.
Notice that memoization is not a space efficient approach to solving subsequence problems, as there is usually a lot of subproblems to store. In the case of longest increasing subsequence, memoization actually has a worse space complexity than the brute force approach.

A sample python implementation is shown in Figure \ref{fig:lis-memo}.

\begin{figure}[H]
    \centering
    \begin{lstlisting}
    def lis_memo(nums):
        if not nums:
            return 0
    
        memo = {}
    
        def dfs(prev_index, current_index):
            if current_index == len(nums):
                return 0
    
            if (prev_index, current_index) in memo:
                return memo[(prev_index, current_index)]
            
            exclude_current = dfs(prev_index, current_index + 1)
    
            include_current = 0
            if prev_index < 0 or nums[current_index] > nums[prev_index]:
                include_current = 1 + dfs(current_index, current_index + 1)
    
            # Save the result in the memoization dictionary
            memo[(prev_index, current_index)] = max(include_current, exclude_current)
    
            return memo[(prev_index, current_index)]
    
        return dfs(-1, 0)
    \end{lstlisting}
    \caption{Longest Increasing Subsequence Memoization Python Implementation}
    \label{fig:lis-memo}
\end{figure}



\subsection{Complexity Analysis of the Memoization Approach to Longest Increasing Subsequence}
Let n be the length of nums.
\begin{description}
    \item[Time Complexity:]
        For each unique combination of $(prev\_index, current\_index)$, the algorithm either calculates the result or looks it up in the $memo$ table.
        Each subproblem is calculated only once.
        The algorithm explores all combinations of $prev\_index$ and $current\_index$.
        There are at most $n$ choices for $current\_index$ and,
        in the worst case, $n$ choices for $prev\_index$ for each $current\_index$.
        Therefore, the total number of unique subproblems is $O(n^2)$.
        
    \item[Space Complexity:] 
        The space complexity is increased to $O(n^2)$,
        as the $memo$ table needs to store all $n^2$ combinations of $prev\_index$ and $current\_index$ in the worst case.
        
        
    \item[Overall:] Total:\\
        Time Complexity: $O(n^2)$\\
        Space Complexity: $O(n^2)$
    
\end{description}

\subsection{Tabulation Approach to Longest Increasing Subsequence}
We can use tabulation to build a table from which we can deduce the result, similar to the Coin Change Problem.
We know that a subsequence which consists of only the last index will result in an increasing subsequence of length 1.
We can work backwards, for the second last, third last ect.. deciding if including that element will result in a longer increasing subsequence or not,
and storing the longest possible increasing subsequence starting at each index until we reach index 0.
We create a table called $dp$ of size $|nums|$, where $dp[i]$ represents the longest increasing subsequence starting at index $i$ in nums.
Lets take the example $nums = [1,2,4,3]$.
We initialize $dp[3] \leftarrow 1$, as the longest increasing subsequence starting at index 3 is 1.
Now, consider $nums[2] = 4$
We can either take $nums[2]$ as a subsequence by itself, or include $nums[2]$ in any subsequence at any index that comes after it (as long as it maintains the property of an increasing subsequence).
Including it would make $dp[2] = 1+dp[3]$,
Excluding it would make $dp[2] = 1$.
Since including it would not result in an increasing subsequence\footnote{The resulting sequence $[4,3]$ is not increasing.}, we must exclude it, so $dp[2] = 1$
Now Conisder $nums[1] = 2$.
We can either take it by itself or include it in any subsequence at any index that comes after it.
Including it would make $dp[1] \leftarrow max(1+dp[2], 1+dp[3])$, and taking it by itself would make $dp[1] = 1$.
We choose the option which maximizes the value of $dp[1]$, which is $1+dp[2]$\footnote{Or, equally, $1+dp[3]$.} $= 2$.
So for $dp[i]$, by the same logic, we simply put $max(1,1+dp[j1],1+dp[j2],1+dp[j3]...)$\footnote{Where $jx$ is the index which comes $x$ places after $i$.}.
(only include $1+dp[jx]$ in the max function if $nums[i] < nums[jx]$, to maintain increasing subsequence property).
This will give us a table where $dp[i]$ contains the longest increasing subsequence of $nums$ where the first number is $nums[i]$.
We can then simply return $max(dp)$.

A sample python implementation is shown in Figure \ref{fig:lis-dp}.

\begin{figure}[H]
    \centering
    \begin{lstlisting}
    def lis_dp(nums):
        dp = [1] * len(nums)
    
        for i in range(len(nums)-1,-1,-1):
            for j in range(i+1,len(nums)):
                if nums[i] < nums[j]:
                    dp[i] = max(dp[i], 1+dp[j])
    
        return max(dp)
    \end{lstlisting}
    \caption{Longest Increasing Subsequence Tabulation Python Implementation}
    \label{fig:lis-dp}
\end{figure}

\subsection{Complexity Analysis of the Tabulation Approach to Longest Increasing Subsequence}
Let n be the length of nums.
\begin{description}
    \item[Time Complexity:]
        For the worst case scenario, we need to perform a double nested iteration over nums.
        All other operations within the loops are constant time, so the time complexity is of order $O(n^2)$.
        
    \item[Space Complexity:] 
        The space complexity is determined by the size of the $dp$ array. This array is always of size $n$.
        Therefore the space complexity is $O(n)$.
        
        
    \item[Overall:] Total:\\
        Time Complexity: $O(n^2)$\\
        Space Complexity: $O(n)$
    
\end{description}
