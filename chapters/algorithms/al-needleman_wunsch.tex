This famous dynamic programming algorithm is used for global alignment of DNA and protein sequences.

\begin{description}
    
    \item[Problem Statement:]
        Given two character sequences $seq1$ and $seq2$,
        place zero or more 'gap's (denoted $"-"$) in $seq1$ and/or $seq2$ such that the maximum number of characters in $seq1$ and $seq2$ are 'aligned'.
        Characters $X$ and $Y$ are 'aligned' when $X == Y$ and the index of $X$ in $seq1$ is exactly equal to the index of $Y$ in $seq2$.
        
    \item[Input:] 
        Two strings $seq1$ and $seq2$.
        
    \item[Output:]
        Two strings $align1$ and $align2$.
        
    \item[Example:] For:\\
        $seq1 = "ATGCT"$\\
        $seq2 = "AGCT"$\\
        $align1 = "ATGCT"$\\
        $align2 = "A-GCT"$

    \item[Explanation:]
        Globally Aligned sequences:\\
        A T G C T\\
        A - \ G C T
\end{description}

The trick is to find an efficient way to place gaps in the sequences to maximise the amount of globally aligned letters.
We can do this with tabulation:
\newpage

Create a matrix of size: $(len(seq1) + 1)$ x $(len(seq2) + 1)$ as follows:
(Note that the visualization of the table includes an additional row and column which contain labels in \textbf{bold} representing the characters of $seq1$ and $seq2$ respectively. These are not part of the actual Needleman-Wunch table and are included for clarity.)

\begin{table}[H]
    \centering
    \begin{tabular}{|c|c|c|c|c|c|c|}
        \hline
          & $\phantom{\textbf{A}}$ & \textbf{A} & \textbf{T} & \textbf{G} & \textbf{C} & \textbf{T} \\
        \hline
         & $\phantom{0}$ &  &  &  &  &  \\
        \hline
        \textbf{A} &  &  &  &  &  &  \\
        \hline
        \textbf{G} &  &  &  &  &  &  \\
        \hline
        \textbf{C} &  &  &  &  &  &  \\
        \hline
        \textbf{T} &  &  &  &  &  &  \\
        \hline
    \end{tabular}
    \caption{Empty Needleman-Wunsch Table.}
\end{table}


Now use the following scheme to fill in the table:

$Match: 1$

$Mismatch: -1$

$GAP: -2$

\subsection{Initialization of the Needleman-Wunsch Table}

Starting with 0 at $(0,0)$, fill the extra row and column with progressive GAP penalties as follows:

\begin{table}[htbp]
    \centering
    \begin{tabular}{|c|c|c|c|c|c|c|}
        \hline
          & $\phantom{\textbf{A}}$ & \textbf{A} & \textbf{T} & \textbf{G} & \textbf{C} & \textbf{T} \\
        \hline
        $\phantom{\textbf{A}}$& 0 & -2 & -4 & -6 & -8 & -10 \\
        \hline
        \textbf{A} & -2 &  &  &  &  &  \\
        \hline
        \textbf{G} & -4 &  &  &  &  &  \\
        \hline
        \textbf{C} & -6 &  &  &  &  &  \\
        \hline
        \textbf{T} & -8 &  &  &  &  &  \\
        \hline
    \end{tabular}
    \caption{Initialized Needleman-Wunsch Table.}
\end{table}

\subsection{Filling the Needleman-Wunsch Table}

Now starting at $(1,1)$, and going row-wise, fill each cell with the max of the following:

\begin{enumerate}
    \item Value from left + $GAP$
    
    \item Value from above + $GAP$
    
    \item Value from diagonal + ($Match$ or $Mismatch$) [depending on wether the row and column label are the same letter]

\end{enumerate}

So, for the $(1,1)$ cell, we fill it with $max(-4,-4,1) = 1$.

And for the $(1,2)$ cell, we fill it with $max(-1,-6,-3) = -1$.

And so on...

Until we get:

\begin{table}[htbp]
    \centering
    \begin{tabular}{|c|c|c|c|c|c|c|}
        \hline
          &  & \textbf{A} & \textbf{T} & \textbf{G} & \textbf{C} & \textbf{T} \\
        \hline
        & 0 & -2 & -4 & -6 & -8 & -10 \\
        \hline
        \textbf{A} & -2 & 1 & -1 & -3 & -5 & -7 \\
        \hline
        \textbf{G} & -4 & -1 & 0 & 0 & -2 & -4 \\
        \hline
        \textbf{C} & -6 & -3 & -2 & -1 & 1 & -1 \\
        \hline
        \textbf{T} & -8 & -5 & -2 & -3 & -1 & 2 \\
        \hline
    \end{tabular}
    \caption{Filled Needleman-Wunsch Table}
\end{table}

\subsection{Traceback in the Needleman-Wunsch Table}

Starting from the bottom-right (which will always be the highest value in the matrix),
continue the following until $(0,0)$ is reached:

\begin{itemize}
    \item If $row$ and $col$ labels match, go diagonally top-left.

    \item Else, go to $max(left,above,diagonal)$.
    
    \item Each time we go diagonally, we can align the $row$ and $col$ labels.

    \item Each time we go left, we put a gap in $seq2$ at that index.

    \item Each time we go up, we put a gap in $seq1$ at that index.

\end{itemize}

Note that the aligned sequences will be written from right to left.

A sample Python implementation is given in Figure \ref{fig:needleman-wunsch}

\begin{figure}[H]
    \centering
    \begin{lstlisting}[basicstyle=\footnotesize]
    def needleman_wunsch(seq1, seq2, match=1, mismatch=-1, gap=-2):
        # Initialize the scoring matrix
        m, n = len(seq2), len(seq1)
        score = [[0] * (n + 1) for _ in range(m + 1)]
    
        # Initialize the first row and column
        for i in range(m + 1):
            score[i][0] = i * gap
        for j in range(n + 1):
            score[0][j] = j * gap
    
        # Fill in the scoring matrix
        for i in range(1, m + 1):
            for j in range(1, n + 1):
                match_mismatch = match if seq2[i - 1] == seq1[j - 1] else mismatch
                diagonal = score[i - 1][j - 1] + match_mismatch
                horizontal = score[i][j - 1] + gap
                vertical = score[i - 1][j] + gap
                score[i][j] = max(diagonal, horizontal, vertical)
    
        # Traceback to find the alignment
        align2, align1 = "", ""
        i, j = m, n
        while i > 0 or j > 0:
            if i > 0 and j > 0 and score[i][j] == score[i - 1][j - 1] + (match if seq2[i - 1] == seq1[j - 1] else mismatch):
                align2 = seq2[i - 1] + align2
                align1 = seq1[j - 1] + align1
                i -= 1
                j -= 1
            elif i > 0 and score[i][j] == score[i - 1][j] + gap:
                align2 = seq2[i - 1] + align2
                align1 = "-" + align1
                i -= 1
            else:
                align2 = "-" + align2
                align1 = seq1[j - 1] + align1
                j -= 1
    
        return align1, align2
    
    sequence1 = "ATGCT"
    sequence2 = "AGCT"
    align1, align2 = needleman_wunsch(sequence1, sequence2)
    \end{lstlisting}
    \caption{The Needleman Wunsch Algorithm Python Implementation}
    \label{fig:needleman-wunsch}
\end{figure}

