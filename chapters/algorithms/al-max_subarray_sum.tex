\begin{description}
    \item[Problem Statement:]
        Given an integer array, return the largest value that a subarray of the array sums to. A subarray is a contiguous subsequence. This means all elements of the subsequence are strictly consecutive in the original sequence.

    \item[Input:]
        An integer array $nums$.
        
    \item[Output:]
        An integer $max\_sum$. 
        
    \item[Example:] For:\\
        $nums =  [-2,1,-3,4,-1,2,1,-5,4]$\\
        $max\_sum = 6$
        
    \item[Explanation:]
        $[4,-1,2,1]$ is the subarray of $nums$ that has the largest sum, 6.
        
\end{description}


\subsection{Brute Force Approach to Max Subarray Sum}
The brute force way to solve this problem is to generate every possible subarray of $nums$
starting with the subarray containing only $nums[0]$, and ending with the entirety of $nums$.
Then sum each subarray and get the maximum of these sums.
Instead of generating and storing every subarray, we can iterate over each subarray in place by iterating over $nums$ with a $start$ pointer marking the start of the subarray,
and iterate over the rest of the array with an $end$ pointer marking the end of the subarray. This reduces the space complexity from O(n) to O(1) (see section \ref{subsec:ca-max-subarray-sum-bf}).
We can improve this further by keeping track of a $current\_sum$ and $max\_sum$ and updating them dynamically as we execute the $end$ loop rather than summing each subarray after it is generated,
reducing the time complexity from $O(n^3)$ to $O(n^2)$ (see section \ref{subsec:ca-max-subarray-sum-bf}).

A sample Python implementation is shown in Figure \ref{fig:max-subarray-sum-bf}.

\begin{figure}[H]
    \centering
    \begin{lstlisting}
    def max_subarray_sum_bf(nums):
        if not nums:
            return 0
    
        n = len(nums)
        max_sum = float('-inf')
    
        for start in range(n):
            current_sum = 0
            for end in range(start, n):
                current_sum += nums[end]
                max_sum = max(max_sum, current_sum)
    
        return max_sum
    \end{lstlisting}
    \caption{Max Subarray Sum Brute Force Python Implementation.}
    \label{fig:max-subarray-sum-bf}
\end{figure}



\subsection{Complexity Analysis of the Brute Force Approach to MSS}\label{subsec:ca-max-subarray-sum-bf}
Let n be the length of nums.
\begin{description}
    \item[Time Complexity:]
        Due to the use of a double nested for loop,
        generating all subarrays of $nums$ has a time complexity of $O(n^2)$.
        All other operations such as adding to the $current\_sum$, resetting the $current\_sum$ and getting the max of $current\_sum$ and $max\_sum$ are constant time.
        If we were to sum each subarray individually rather than keeping a $current\_sum$, which would be an $O(n)$ operation, the total complexity would be increased to $O(n^3)$.

    \item[Space Complexity:] 
        All other variables stored such as $max\_sum$ and $current\_sum$ are constant space.
        The number of variables stored does not increase as $n$ increases, hence the space complexity is $O(1)$.
        If instead of iterating over each subarray in place, we stored the current subarray separately, the space complexity would be increased to $O(n)$.

    \item[Overall:] Total:\\
        Time Complexity: $O(n^2)$\\
        Space Complexity: $O(1)$
    
\end{description}

    
\subsection{Kadanes Algorithm for Max Subarray Sum}
    
There is a way to find the max subarray sum in a single iteration of $nums$.
This is done using the famous Kadane's algorithm \cite{kadane2023two}.
Kadane's algorithm is shown in Algorithm \ref{algo:kadanes}: 

\begin{algorithm}[H]
    \caption{Kadane's Algorithm}
    \label{algo:kadanes}
    \KwIn{An integer array $nums$.}
    \KwOut{An integer $max\_sum$, the max subarray sum of $nums$.}
    $max\_sum \leftarrow 0$\;
    $current\_sum \leftarrow 0$\;
    \ForEach{$i \in nums$}{
        \If{$current\_sum < 0$}{
            $current\_sum \leftarrow 0$\;
        }
        $current\_sum += i$\;
        \If{$current\_sum > max\_sum$}{
            $max\_sum \leftarrow current\_sum$\;
        }
    }
    \KwRet{$max\_sum$}
\end{algorithm}
\subsection*{Explanation of Kadanes Algorithm}
In Kadane's algorithm, we iterate over $nums$, and at each step increment $current\_sum$ by the value of the current element. 
If $current\_sum$ becomes negative, $current\_sum$ is reset to zero. 
The $max\_sum$ is updated to $current\_sum$ whenever $current\_sum$ becomes greater than $max\_sum$.
If the array consists entirely of negative numbers, the algorithm will return 0 for the maximum subarray sum.
If you want to modify the algorithm such that empty subarrays are not allowed, you can initialize $max\_sum$ and $current\_sum$ to the first element of the array instead of 0.
This way, the algorithm will return the largest single negative element if the array consists entirely of negative numbers.
Kadane's algorithm is considered a dynamic programming algorithm even though it does not use a table, because it satisfies the necessary criteria:

\textbf{Optimal Substructure:} The problem of finding the maximum subarray can be divided into smaller subproblems, where the solution to the problem at each index depends on the solution to the problem at the previous index.

\textbf{Overlapping Subproblems:} The subproblems in Kadane's algorithm overlap, as the solution to the problem at each index relies on the solution to the problem at the previous index.

A sample Python implementation is shown in Figure \ref{fig:kadanes}.

\begin{figure}[H]
    \centering
    \begin{lstlisting}
    def max_subarray_sum_kadanes(nums):
        if not nums:
            return 0
    
        max_sum = current_sum = nums[0]
    
        for num in nums[1:]:
            if current_sum < 0:
                current_sum = 0
            current_sum += num
            max_sum = max(max_sum, current_sum)
    
        return max_sum
    \end{lstlisting}
    \caption{Kadane's Algorithm Python Implementation}
    \label{fig:kadanes}
\end{figure}



\subsection{Complexity Analysis of Kadane's Algorithm}
Let n be the length of nums.
\begin{description}
    \item[Time Complexity:]
        This algorithm consists of a single iteration of nums, which is $O(n)$.
        All other operations such as updating the $current\_sum$, $max\_sum$ are constant time.
        
    \item[Space Complexity:] 
        All variables stored such as $max\_sum$ and $current\_sum$ are constant space,
        and we do not store more information as $n$ grows. Hence the space complexity is $O(1)$.

        
    \item[Overall:] Total:\\
        Time Complexity: $O(n)$\\
        Space Complexity: $O(1)$
    
\end{description}
\newpage