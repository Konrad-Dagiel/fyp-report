We can see that extending a K-Dimensional pathfinding problem where no negative progress is allowed (similar to Unique Paths or Min Path Sum),
we can extend it to a K+1-Dimensional problem by simply considering a single new 'direction' in our calculation.
Because of this, we can solve any pathfinding problem from any corner $R$ of a rank $N$ tensor to the opposite corner $F$ (where no negative progress is allowed),
by simply initializing an $N$ dimensional table $dp$, initializing $dp[F]$ and each field on a border in $dp$ to an appropriate value\footnote{1 in the case of Unique Paths, or $cost[i]$ in Min Path Sum}, and deducing the value of an arbitrary field $v$ by doing a constant time operation (such as $sum$ or $max$) of each 'surrounding' value in a direction opposite to the direction you are allowed to travel in that dimension.


\subsection{N-Dimensional Unique Paths Python Implementation Details}
This implemetation is tricky as we must simulate $n$ for loops where $n$ is the dimension count.
Numpy was used to access, store data in, and manipulate the input tensor as well as the $dp$ table.
This choice was made because numpy allows us to access elements in n-rank tensors by providing a tuple of length $n$ as an index.
Numpy also allows us to easily initialialize n-rank tensors through the use of its $np.zeros(shape)$ function, rather than having to use something similar to Figure \ref{fig:zeros}:

\begin{figure}[H]
    \centering
    \begin{lstlisting}
    def initialize_array(dimensions):
        if len(dimensions) == 1:
            return [0] * dimensions[0]
        else:
            return [initialize_array(dimensions[1:]) for _ in range(dimensions[0])]
    \end{lstlisting}
    \caption{Example code to generate an n-rank tensor of zeros without Numpy}
    \label{fig:zeros}
\end{figure}

Numpy's $np.ndindex()$ is a function that provides an iterator yielding tuples of indices for a given shape.
It is particularly useful for iterating over the indices of n-dimensional arrays.
This function returns an object that can be iterated over, generating all possible index tuples for a specified shape.

The pseudocode of the algorithm using numpy is as follows:

\begin{enumerate}
    \item Initialize an n-dimensional array of zeros called $dp$ to store the number of paths for each cell, with $dp[R]$ set to 1.
    \item The $np.ndindex(shape)$ generates an iterator over all possible indices in the n-dimensional array.
    \item For each index, represented by the $current\_cell$:
    \item Iterate over each dimension using for $i$ in $range(num_dimensions)$.
    \item Check if the current cell's index in the current dimension ($current\_cell[i]$) is greater than 0. If true, it means there is a valid cell to move from in that dimension.
    \item Create a copy of the current cell (denoted $prev\_cell$) and decrement the index in the current dimension ($prev\_cell[i] -= 1$). This represents the cell from which we are coming.
    \item Add the number of paths from the previous cell to the current cell in the $dp$ array ($dp[index] += dp[tuple(prev\_cell)]$).

\end{enumerate}

\subsection{Python Implementation of N-Dimensional Unique Paths}

Figure \ref{fig:nd-unique-paths} shows a Python Implementation of N-Dimensional Unique Paths using the framework discussed above.
The code is heavily commented so we can see the framework in action. Note that this approach can be trivially converted to calculate the Min Path Sum, or any other pathfinding problem where negative progress is not allowed.

\begin{figure}[H]
    \centering
    \begin{lstlisting}
    import numpy as np

    def unique_paths_nd(dimensions):
        # Determine the number of dimensions
        num_dimensions = len(dimensions)
    
        # Initialize an n-dimensional array to store the number of paths for each cell
        shape = tuple(dimensions)
        dp = np.zeros(shape, dtype=int)
    
        # Set the number of paths for the starting cell to 1
        dp[(0,) * num_dimensions] = 1
    
        # Calculate the number of paths for each cell in the matrix
        for index in np.ndindex(shape):
            # Get the current index as an array
            current_cell = np.array(index)
            # Iterate over the array
            for i in range(num_dimensions):
                # If the current cell is not on the edge
                if current_cell[i] > 0:
                    # Add the previous cell's paths to the current cell [this happens from each valid direction in each dimension]
                    prev_cell = current_cell.copy()
                    prev_cell[i] -= 1
                    dp[index] += dp[tuple(prev_cell)]
    
        # The result is stored in the last cell of the matrix
        return dp[tuple(np.array(dimensions) - 1)]
    
    # Example usage:
    dimensions = (2,7,5)
    print(unique_paths_nd(dimensions))
    \end{lstlisting}
    \caption{Python Implementation of N-Dimensional Unique Paths}
    \label{fig:nd-unique-paths}
\end{figure}

