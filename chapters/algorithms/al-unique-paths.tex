\begin{description}
    \item[Problem Statement:]
        A robot who's location on a grid is denoted $R$ is located at the top-left corner of a $m$ x $n$ grid.
        The robot is trying to reach the finish square at the bottom right corner of the grid donoted as $F$.
        How many possible unique paths to $F$ exist starting at $R$, given the constraint that the robot can only move right one square, or down one square at any given point in time?
        
    \item[Input:]
        Two integers $m$ and $n$, which are the dimensions of the grid.
        
    \item[Output:] 
        An integer $unique\_paths(m,n)$.
        
    \item[Example:] For:\\
        $m = 3$, $n = 7$\\
        $unique\_paths(m,n) = 28$
        
    \item[Explanation:]
        There are 28 unique ways the robot can reach $F$ from $R$.
        Example Path:

        \begin{table}[htbp]
            \centering
            \begin{tabular}{|c|c|c|c|c|c|c|}
                \hline
                \textbf{R} & $\rightarrow$ & $\downarrow$ &  &  &  &  \\
                \hline
                 &  & $\rightarrow$ & $\rightarrow$ & $\rightarrow$ & $\rightarrow$ & $\downarrow$ \\
                \hline
                 &  &  &  &  &  & \textbf{F} \\
                \hline
            \end{tabular}
            \caption{Example Path for Unique Paths}
        \end{table}

        
\end{description}

\subsection{Tabulation Approach to Unique Paths}
Notice that wherever the robot lands on the grid, it is possible to reach the finish square,
so we do not have to worry about cases where the robot cannot reach the finish square (backtracking).
Notice also that the number of paths from any square is the sum of the number paths to the right, and the number of paths if you go down.
We can use tabulation to solve the problem, by creating an $m$ x $n$ table called $dp$ where the value at $dp[i][j]$ represents the number of unique paths from that square to $F$.
The value at $dp[0][0]$ will hence contain the solution to the problem.
We can then initialize $dp[F]$ to 1, as there is one path from F to itself (the path contains zero moves).
The square directly above and to the left of $dp[F]$ can also be initialized to 1, as there is one path from them to $F$, going down or right respectively. This logic follows for all of the bottom row of the grid, and the last column of the grid.

\begin{table}[H]
    \centering
    \begin{tabular}{|c|c|c|c|c|c|c|}
        \hline
         &  &  &  &  &  & 1 \\
        \hline
         &  &  &  &  &  & 1 \\
        \hline
        1 & 1 & 1 & 1 & 1 & 1 & 1 \\
        \hline
    \end{tabular}
    \caption{Example Initialized $dp$ Table for Unique Paths}
\end{table}

Now we simply fill the grid from the bottom right to the top left,
where the value of each square is the sum of the value below it and to its right.

\begin{table}[H]
    \centering
    \begin{tabular}{|c|c|c|c|c|c|c|}
        \hline
        28 & 21 & 15 & 10 & 6 & 3 & 1 \\
        \hline
        7 & 6 & 5 & 4 & 3 & 2 & 1 \\
        \hline
        1 & 1 & 1 & 1 & 1 & 1 & 1 \\
        \hline
    \end{tabular}
    \caption{Example Filled $dp$ Table for Unique Paths}
\end{table}

The top left square is where the robot was, so that is the number of paths from the robot to the finish square.

A sample Python implementation is given in \ref{fig:unique-paths}
\begin{figure}[H]
    \centering
    \begin{lstlisting}
    def unique_paths(m,n):
    
        # Initialize a m x n table of 1s
        dp = [[1] * n for _ in range(m)]
    
        # Fill in the table in a bottom-up manner
        for i in range(m-1,-1,-1):
            for j in range(n-1,-1,-1):
                # Leave last row and column as 1s
                if i == m-1 or j == n-1:
                    continue
                # Fill all other squares with the sum of the square below and to the right
                else:
                    dp[i][j] = dp[i+1][j] + dp[i][j+1]

        return dp[0][0]
    \end{lstlisting}
    \caption{Unique Paths Python Implementation}
    \label{fig:unique-paths}
\end{figure}

\subsection{Complexity Analysis of Unique Paths}\label{ca-unique-paths}

\begin{description}
    \item[Time Complexity:]
        The time complexity of filling an $m$ x $n$ table with values,
        where each value is calculated by two constant time lookups and a constant time addition, is $O(m * n)$.
        
    \item[Space Complexity:] 
        The space complexity of storing an $m$ x $n$ table where each field in the table contains a single integer is $O(m * n)$.

    \item[Overall:] Total:\\
        Time Complexity: $O(m * n)$\\
        Space Complexity: $O(m * n)$
    
\end{description}
\newpage

\subsection{Optimization of the Unique Paths Problem}
Notice that for a given value in row $r$ in the table $dp$, $v$ can be calculated using just row $r$ and row $r-1$.
Since we compute the values in $dp$ rowwise from the bottom up, we do not actually need to store the entire $dp$ table in memory as the values which are two or more rows below the row we are computing at any given time do not contribute to our calculation.
Storing only two rows of the $dp$ table at a time reduces the space complexity from $O(m * n)$ to just $O(m)$.\\

A Python implementation of this optimization is given in Figure \ref{fig:unique-paths-optimized}.

\begin{figure}[H]
    \centering
    \begin{lstlisting}
    def unique_paths_optimized(m,n):
        # Initialize the bottom row of 1s
        oldRow = [1] * n
    
        # For each subsequent row in the grid
        for _ in range(m-1):
            # Create a new row which is initialized to all 1s
            newRow = [1] * n
            # Traverse the new row in reverse, without the last element
            for j in range(n - 2, -1, -1):
                # Each value in the new row is the sum of the value to the right and below
                newRow[j] = newRow[j+1] + oldRow[j]
            oldRow = newRow
    
        return oldRow[0]
    \end{lstlisting}
    \caption{Unique Paths Optimized Python Implementation}
    \label{fig:unique-paths-optimized}
\end{figure}

A further optimization involves swapping the values $m$ and $n$ such that $m > n$. We will arrive at the same answer, but storing smaller rows in memory.
This optimization further reduces the space complexity of Unique Paths to $O(min(m,n))$.

We have seen a similar optimization in Section \ref{subsec:lcs-optimized}.
\newpage