\begin{description}
    \item[Problem Statement:]
        Given a 2D array filled with non-negative numbers representing the 'cost' at that field in the input matrix,
        find the cost of the path from the top-left corner (denoted $R$) to the bottom-right corner (denoted $F$) which minimizes the sum of numbers along the path.
        (minimizes path cost).
        You can only move down or to the right at any point in time (no path can make negative progress).

    \item[Input:]
        A 2D array, called $cost$.
        
    \item[Output:]
        An integer $min_path_sum(cost)$, which represents the cost of the minimum path from $R$ to $F$.
        
    \item[Example:] For:\\
    $cost = $\\
    $[[1,3,1],$\\
    $[1,5,1],$\\
    $[4,2,1]]$\\

    $min_path_sum(cost) = 7$

    \item[Explanation:]
        Explanation: $1 + 3 + 1 + 1 + 1 = 7$ is the minimum path sum.
        \begin{table}[H]
            \centering
            \begin{tabular}{|c|c|c|}
                \hline
                \textbf{R} & $\rightarrow$ & $\downarrow$ \\
                \hline
                 &  & $\downarrow$ \\
                \hline
                 &  & \textbf{F} \\
                \hline
            \end{tabular}
        \end{table}
        
\end{description}

\subsection{Tabulation Approach to Minimum Path Sum}
We can solve this problem in a very similar manner to Unique Paths.
We start by creating a table $dp$ with the same dimensions as the input matrix, where $dp[i][j]$ represents the minimum path cost from $dp[i][j]$ to $F$.
We can initialize $F$ to be it's own cost in the input matrix (the cost of moving from $F$ to $F$ is $cost[F]$).

\begin{table}[H]
    \centering
    \begin{tabular}{|c|c|c|}
        \hline
         &  &  \\
        \hline
         &  &  \\
        \hline
         &  & 1 \\
        \hline
    \end{tabular}
\end{table}
Now we can iterate through the input matrix backwards using the same logic as in Unique Paths, but this time:


\begin{enumerate}
    \item If we are on the last column: $dp[i][j] = dp[i][j+1] + cost[i][j]$.
    [This is because there is no more squares to go right, so we must go down and incur that cost]

    \item If we are on the last row: $dp[i][j] = dp[i+1][j] + cost[i][j]$.
    [This is because there is no more squares to go down, so we must go right and incur that cost]

    \item Otherwise, $dp[i][j] = cost[i][j] +$ the minimum cost of going down, or going right.

\end{enumerate}

The top left of the grid will give us the minimum cost of reaching the bottom right position.

\begin{table}[H]
    \centering
    \begin{tabular}{|c|c|c|}
        \hline
        7 & 6 & 3 \\
        \hline
        8 & 7 & 2 \\
        \hline
        7 & 3 & 1 \\
        \hline
    \end{tabular}
\end{table}

A sample Python Implementation of this is given in Figure \ref{fig:min-path-sum}

\begin{figure}[H]
    \centering
    \begin{lstlisting}
    def min_path_sum(cost):
        rows, cols = len(cost), len(cost[0])
        
        # Initialize a table to store minimum path sums
        dp = [[0] * cols for _ in range(rows)]
    
        # Fill in the table in a bottom-up manner
        for i in range(rows-1,-1,-1):
            for j in range(cols-1,-1,-1):
                # Initialize the bottom right square to its own cost
                if i == rows-1 and j == cols-1:
                    dp[i][j] = cost[i][j]
    
                # When we are on the last row or col
                elif i == rows-1:
                    dp[i][j] = dp[i][j+1] + cost[i][j]
                elif j == cols-1:
                    dp[i][j] = dp[i+1][j] + cost[i][j]
    
                # Otherwise, choose the minimum cost path and add its cost
                # to the cost of this square.
                else:
                    dp[i][j] = cost[i][j] + min(dp[i+1][j], dp[i][j+1])
    
        return dp[0][0]
    \end{lstlisting}
    \caption{Min Path Sum Python Implementation}
    \label{fig:min-path-sum}
\end{figure}

\subsection{Complexity Analysis of Min Path Sum}
Let $m$ be the number of rows in $cost$ (the input matrix), and $n$ be the number of columns in $cost$.

\begin{description}
    \item[Time Complexity:]
        The time complexity of filling an $m$ x $n$ table with values,
        where each value is calculated by two constant time lookups and a constant time addition, is $O(m * n)$.
        
    \item[Space Complexity:] 
        The space complexity of storing an $m$ x $n$ table where each field in the table contains a single integer is $O(m * n)$.
        
    \item[Overall:] Total:\\
        Time Complexity: $O(m * n)$\\
        Space Complexity: $O(m * n)$
    
\end{description}

\subsection{Optimization of Minimum Path Sum}
We can use the exact same optimization as we did with Unique Paths, as each value in row $r$ can be calculated from just the values in row $r$ and row $r-1$.
Storing only two rows at a given time will reduce the Space complexity to $O(m)$.

A sample Python implementation is given in Figure \ref{fig:min-path-sum-optimized}
\begin{figure}[H]
    \centering
    \begin{lstlisting}
    def min_path_sum_optimized(cost):
        rows, cols = len(cost), len(cost[0])
        
        # Initialize the bottom row of 0s
        oldRow = [0] * cols
    
        # For each subsequent row in the grid
        for i in range(rows-1, -1, -1):
            # Create a new row which is initialized to all 0s
            newRow = [0] * cols
            # Traverse the new row in reverse, same logic as before.
            for j in range(cols - 1, -1, -1):
                if i == rows-1 and j == cols-1:
                    newRow[j] = cost[i][j]
                elif i == rows-1:
                    newRow[j] = newRow[j+1] + cost[i][j]
                elif j == cols-1:
                    newRow[j] = oldRow[j] + cost[i][j]
                else:
                    newRow[j] = cost[i][j] + min(oldRow[j], newRow[j+1])
            oldRow = newRow
    
        return oldRow[0]
    \end{lstlisting}
    \caption{Min Path Sum Optimized Python Implementation}
    \label{fig:min-path-sum-optimized}
\end{figure}
\subsection{Note on Optimization}
This two-row framework can be used to solve any pathing problem which is constrained such that negative progress is not possible.
This optimization can be taken even further. We have solved both problems rowwise starting from the bottom. This problem can equally be solved columnwise,
using the same optimization except this time storing two columns instead of two rows.
If we solve these problems rowwise in the case of $m \leq n$, and columnwise if $m > n$, we actually reduce the space complexity even further to $O(min(m,n))$.