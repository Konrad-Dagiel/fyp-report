\begin{description}
    \item[Problem Statement:]
        Given a string, return the longest palindromic substring\footnote{Substrings must be contiguous.} of the string.

    \item[Input:] 
        A string $s$.
        
    \item[Output:] 
        A string $lpcs(s)$, which is the longest palindromic substring of $s$.
        
    \item[Example:] For:\\
        $s = "aaaabbaa"$\\
        $lpcs(s) = "aabbaa"$

    \item[Explanation:]
        The longest palindromic substring of $"aaaabbaa"$ is $"aabbaa"$.
        Note that here we are asked for the substring itself, rather than the length of the substring.
   
\end{description}

\subsection{Brute Force Approach to Longest Contiguous Palindromic Substring}
If we were to tackle this problem the brute force way, we would have to iterate over all substrings of s, filtering out non palindromes along the way,
and keeping track of the longest palindromic substring found so far. 
We would then return the longest palindromic substring of s.

\subsection{Complexity Analysis of the Brute Force Approach to Longest Contiguous Palindromic Substring}
Let n be the length of s.

\begin{description}
    \item[Time Complexity:]
        Iterating over all substrings of a string of length $n$ has a time complexity of $O(n^2)$.
        Checking if a substring is a palindrome has a time complexity of $O(n)$.
        This gives the brute force approach an overall time complexity of $O(n^3)$.
        
    \item[Space Complexity:] 
        The space complexity of generating all substrings and storing them in a list is also $O(n^3)$
        because we have $O(n^2)$ substrings, each of which has a maximum length of $O(n)$.
        However, in our algorithm it is possible to iterate over the subsrtings of $s$ in place using a double for loop
        \footnote{See Figure \ref{fig:max-subarray-sum-bf} for an example of how to do this.}, we can store one substring at a time,
        which has a maximum length of $n$. This brings the space complexity down to $O(n)$.
        
    \item[Overall:] Total:\\
        Time Complexity: $O(n^3)$\\
        Space Complexity: $O(n)$
    
\end{description}
Notice that if a string $p$ is a palindrome, and $x$ is a character, the string $xpx$ is guaranteed to be a palindrome.
This is proof of optimal substructure.
The problem of determining whether a substring is a palindrome or not can be reduced to smaller subproblems.
For example, when checking if $s[i:j]$ is a palindrome, we often need to check if $s[i+1:j-1]$ is a palindrome,
which overlaps with other similar subproblems.
Therefore, we can use dynamic programming principles to optimize the solution to this problem.
\subsection{Tabulation Approach to Longest Contiguous Palindromic Substring.}
We can create a table $dp$ where $dp[i][j]$ is 1 if $s[i:j+1]$ is a palindrome, else 0. (eg: if $dp[1][3] = 1$, $s[1:4] (\text{"aaa" in our example string})$ is a palindrome.)
We start with a table $dp$ which is a 2D matrix of size $|s| * |s|$.
We can initialize all fields where $i == j$ to 1, as single letters are palindromes.

\begin{table}[htbp]
    \centering
    \begin{tabular}{|c|c|c|c|c|c|c|c|c|}
        \hline
          & \textbf{a} & \textbf{a} & \textbf{a} & \textbf{a} & \textbf{b} & \textbf{b} & \textbf{a} & \textbf{a} \\
        \hline
        \textbf{a} & 1 &  &  &  &  &  &  &  \\
        \hline
        \textbf{a} &  & 1 &  &  &  &  &  &  \\
        \hline
        \textbf{a} &  &  & 1 &  &  &  &  &  \\
        \hline
        \textbf{a} &  &  &  & 1 &  &  &  &  \\
        \hline
        \textbf{b} &  &  &  &  & 1 &  &  &  \\
        \hline
        \textbf{b} &  &  &  &  &  & 1 &  &  \\
        \hline
        \textbf{a} &  &  &  &  &  &  & 1 &  \\
        \hline
        \textbf{a} &  &  &  &  &  &  &  & 1 \\
        \hline
    \end{tabular}
\end{table}

We can initialize all fields where $j = i+1$ and $s[i] = s[j]$ to 1 as all pairs of the same letter are palindromes.
This handles the case where the palindrome has an even amount of characters (meaning the center of the palindrome is a character pair).


\begin{table}[htbp]
    \centering
    \begin{tabular}{|c|c|c|c|c|c|c|c|c|}
        \hline
          & \textbf{a} & \textbf{a} & \textbf{a} & \textbf{a} & \textbf{b} & \textbf{b} & \textbf{a} & \textbf{a} \\
        \hline
        \textbf{a} & 1 & 1 &  &  &  &  &  &  \\
        \hline
        \textbf{a} &  & 1 & 1 &  &  &  &  &  \\
        \hline
        \textbf{a} &  &  & 1 & 1 &  &  &  &  \\
        \hline
        \textbf{a} &  &  &  & 1 & 0 &  &  &  \\
        \hline
        \textbf{b} &  &  &  &  & 1 & 1 &  &  \\
        \hline
        \textbf{b} &  &  &  &  &  & 1 & 0 &  \\
        \hline
        \textbf{a} &  &  &  &  &  &  & 1 & 1 \\
        \hline
        \textbf{a} &  &  &  &  &  &  &  & 1 \\
        \hline
    \end{tabular}
\end{table}

Now, notice that if a string $p$ is a palindrome, and $x$ is a character, the string $xpx$ is guaranteed to be a palindrome.
Using this, for all substrings $s'$ of length 3, we check if the start and end are the same letter, and the middle is a palindrome (we know from the existing entries in the table).
If so, we know that $s'$ is a palindrome, so we can set $dp[i][j]$ to 1.
We can put a 1 in $dp[i][j]$ the table if and only if:

\begin{enumerate}
    \item If $s[i] == s[j]$ (the starting character is equal to the ending character).

    AND

    \item If $dp[i+1][j-1]$ == 1 (the string $p$ in between $i$ and $j$ is already known to be a palindrome).

\end{enumerate}

We will end up with a table as follows:

\begin{table}[htbp]
    \centering
    \begin{tabular}{|c|c|c|c|c|c|c|c|c|}
        \hline
          & \textbf{a} & \textbf{a} & \textbf{a} & \textbf{a} & \textbf{b} & \textbf{b} & \textbf{a} & \textbf{a} \\
        \hline
        \textbf{a} & 1 & 1 & 1 &  &  &  &  &  \\
        \hline
        \textbf{a} &  & 1 & 1 & 1 &  &  &  &  \\
        \hline
        \textbf{a} &  &  & 1 & 1 & 0 &  &  &  \\
        \hline
        \textbf{a} &  &  &  & 1 & 0 & 0 &  &  \\
        \hline
        \textbf{b} &  &  &  &  & 1 & 1 & 0 &  \\
        \hline
        \textbf{b} &  &  &  &  &  & 1 & 0 & 0 \\
        \hline
        \textbf{a} &  &  &  &  &  &  & 1 & 1 \\
        \hline
        \textbf{a} &  &  &  &  &  &  &  & 1 \\
        \hline
    \end{tabular}
\end{table}

Do this for all lengths from $3 \rightarrow len(s) - 1$.
We will end up with the following table.

\begin{table}[htbp]
    \centering
    \begin{tabular}{|c|c|c|c|c|c|c|c|c|}
        \hline
          & \textbf{a} & \textbf{a} & \textbf{a} & \textbf{a} & \textbf{b} & \textbf{b} & \textbf{a} & \textbf{a} \\
        \hline
        \textbf{a} & 1 & 1 & 1 & 1 & 0 & 0 & 0 & 0 \\
        \hline
        \textbf{a} &  & 1 & 1 & 1 & 0 & 0 & 0 & 0 \\
        \hline
        \textbf{a} &  &  & 1 & 1 & 0 & 0 & 0 & 1 \\
        \hline
        \textbf{a} &  &  &  & 1 & 0 & 0 & 1 & 0 \\
        \hline
        \textbf{b} &  &  &  &  & 1 & 1 & 0 & 0 \\
        \hline
        \textbf{b} &  &  &  &  &  & 1 & 0 & 0 \\
        \hline
        \textbf{a} &  &  &  &  &  &  & 1 & 1 \\
        \hline
        \textbf{a} &  &  &  &  &  &  &  & 1 \\
        \hline
    \end{tabular}
\end{table}


From the table we can deduce all palindromic substrings, including the longest palindromic substring.
To get the longest palindromic substring from the table, we track $start$ and $max\_length$, which get updated every time a new palindrome is found.
This works because palindromes are found from shortest to longest, so every new palindrome found is the maximum length palindrome.
We can then simply return a slice of the input string as follows: $s[start,start+max_length]$.
This will yield the maximum length palindromic substring.

A sample python implementation is shown in Figure \ref{fig:lpcs-dp}.

\begin{figure}[H]
    \centering
    \begin{lstlisting}
    def lpcs(s):
        n = len(s)
        # Single character strings are palindromes, so we can return s
        if n <=1: return s
    
        # Create a n*n table of zeros
        dp = [[0] * n for _ in range(n)]
    
        # Initialize all single character substrings to 1

        # Single character substrings start and end at the same index, so we locate them with dp[i][i]
        for i in range(n):
            dp[i][i] = 1
    
        # Since we are looking for the longest palindromic substring itself and not the length,
        # we can track the starting position and max_length for convenience
        start = 0
        max_length = 1
    
        # Initialize all pairs of identical characters to 1
        for i in range(n-1):
            if s[i] == s[i+1]:
                dp[i][i+1] = 1
                start, max_length = i,2
    
        # For all substrings with length >=3, starting at length 3, 
        # set dp[i][j] to 1 if start and end are the same letter
        # and the middle is a palindrome
                
        for length in range(3, n+1):
            for i in range(n - length + 1):
                j = i + length - 1
                if dp[i+1][j-1] and s[i] == s[j]:
                    dp[i][j] = 1
                    start, max_length = i, length
    
        return s[start:start+max_length]
    \end{lstlisting}
    \caption{Longest Contiguous Palindromic Substring Tabulation Python Implementation}
    \label{fig:lpcs-dp}
\end{figure}

\subsection{Complexity Analysis of the Tabulation Approach to Longest Contiguous Palindromic Subsequence}
Let $n$ be the length of $s$.

\begin{description}
    \item[Time Complexity:]
        The time complexity to build an $n * n$ table,
        where the values are deduced from a constant time check and a constant time lookup is $O(n^2)$. 

    \item[Space Complexity:] 
        The space complexity to store an $n * n$ table is $O(n^2)$.

        
    \item[Overall:] Total:\\
        Time Complexity: $O(n^2)$\\
        Space Complexity: $O(n^2)$
    
\end{description}