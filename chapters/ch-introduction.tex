\chapter{Introduction}
\pagenumbering{arabic}
\setcounter{page}{1}
Dynamic programming is a powerful technique widely used in computer science for solving optimization and counting problems efficiently.
This project aims to address the challenge of creating an online course with the goal of teaching the dynamic programming technique in a way which can be easily digested by an average undergraduate computer science student.
The report contains detailed descriptions of eleven famous dynamic programming algorithms,
structured in a format which can be used as teaching material by a lecturer looking to give a class on dynamic programming to students with no prior knowledge of the technique.
Each of the problems include a Python implementation of the brute force, memoization and tabulation approach to the problem, a comprehensive and easily digestable complexity analysis of each of the approaches, as well as a rigorous comparative analysis of the runtimes of each of the approaches.
The code is accessible online for students looking to run it on their own inputs, and can be set to print the table it uses to arrive at the solution.
Finally, the project proposes a solution for two new dynamic programming problems, Python implementations of solutions to these problems, complexity analysis of these implementations, and a discussion of potential optimizations which could be made to make these solutions even more efficient.
